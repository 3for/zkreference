\section{Specifying Statements for ZK}
\label{security:spec-statements-ZK}

 
	This document considers types of statements defined by a relation $R$ between instances $x$ and witnesses $w$.
	The relation $R$ specifies which pairs $(x,w)$ are considered related to each other, and which are not related to each other.
	The relation defines a matching language $L$ consisting of instances $x$ that have a witness $w$ in $R$.


	\revblock[rev:pok:types-of-statement]{\ref{it:pok:types-of-statement}}
	A \emph{statement} is either a \emph{membership} claim of the form ``$x \in L$'', 
or a \emph{knowledge} claim of the form ``In the scope of relation $R$, I know a witness for instance $x$.''
	For some cases, the \emph{knowledge} and \emph{membership} types of statement can be informally considered interchangeable, but formally there are technical reasons to distinguish between the two notions. 
	In particular, there are scenarios where a statement of knowledge cannot be converted into a statement of membership, and vice-versa (as exemplified in Section 1.4).
	The examples in this document are often based on statements of knowledge.
	\loosen

 
	The relation $R$ can for instance be specified as a program (e.g. in C or Java), which given inputs $x$ and $w$ decides to accept, meaning $(x,w) \in R$, or reject, meaning $w$ is not a witness to $x \in L$. 
	Examples of such specifications of the relation are detailed in the \hyperref[chap:apps]{Applications track}.
	In the academic literature, relations are often specified either as 
random access memory (RAM) programs or through Boolean and arithmetic circuits, which we describe below.



%%%%%%%%%%%%%%%%%%%%%%%%%%%%%%%%%%%%%%%%%%%%%%%%%%%%%%%%%%%%%%%%
\subsection{Circuit representation}
\label{security:spec-statements-ZK:circuit-representation}

\revblock[rev:chap-security:circuits-vs-R1cs:new-subsection:circuit-representation]{\ref{it:circuits-vs-R1CS:new-subsections}}
A circuit is a directed acyclic graph (DAG) comprised of nodes and labels for nodes, which satisfy the following constraints:
\begin{itemize}
\item Nodes with in-degree 0 are referred to as the \textbf{input nodes}\luissug{Consider adjusting all items to become of the form \textbf{<new term>}: <description>.} and are labeled with some constant (e.g., $0, 1, \ldots$) or with input variable names (e.g., $v_1, v_2, \ldots$)
\item There is a single node with out-degree 0 that is referred to as the \textbf{output node}.
\item Internal nodes are referred to as \textbf{gate nodes} and describe a computation performed at the node.\loosen
\end{itemize}


%%%LB: Consider moving the `parametes' up, because it defines n_x and n_w, which are used in the subsequent bullets
\paragraph{Parameters.} 
Depending on the application, various parameters may be important, for instance the number of gates in the circuit, the number of instance variables $n_x$, the number of witness variables $n_w$, the circuit depth, or the circuit width.
 
\paragraph{Boolean Circuit satisfiability.} 
	The relation $R$ has instances of the form $x = (C, v_1, \ldots, v_{n_x})$ and witnesses $w = (w_1,...,w_{n_w})$. 
	For $(x,w)$ to be in the relation, $C$ must be a circuit with fan-in 2 gate nodes 
that are labeled with Boolean operations, e.g., \XOR\ or \AND, $v_1,...,v_{n_x}$ must specify 
truth values for some of the input nodes, and $w_1,...,w_{n_w}$ must specify 
truth values for the remaining input variables, such that when evaluating the circuit the output node becomes 1 (true).
 
\paragraph{Arithmetic Circuit satisfiability.} 
	The relation has instances of the form $x = (F, C, v_1,...,v_{n_x})$ and witnesses $w = (w_1,...,w_{n_w})$.
	For $(x,w)$ to be in the relation, $F$ must be a finite field (e.g., integers modulo a prime $p$), $C$ must be a circuit with gate nodes that are labeled with field operations, i.e., addition or multiplication, $v_1,...,v_{n_x}$ must specify field elements for some of the input nodes, and $w_1,...,w_{n_w}$ must specify field elements for the remaining input variables, such that when evaluating the circuit the output node becomes 1.
 


%%%%%%%%%%%%%%%%%%%%%%%%%%%%%%%%%%%%%%%%%%%%%%%%%%%%%%%%%%%%%%%%
\subsection{R1CS representation}
\label{security:spec-statements-ZK:R1CS-representation}

\newcommand{\brkt}[1]{\ensuremath{\left\langle #1\right\rangle}}

\revblock[rev:chap-security:circuits-vs-R1cs:new-subsection:R1CS-representation]{\ref{it:circuits-vs-R1CS:new-subsections}}

	A rank-1 constraint system (R1CS) is a system of equations represented by a 
list of triplets $(\vec{a}, \vec{b}, \vec{c})$ of vectors of elements of some field.
	Each triplet defines a ``constraint'' as an equation of the form $(A) \cdot (B) - (C) = 0$.
	Each of the three elements --- (A), (B), (C) --- in such equation is a linear combination 
(e.g., $(C) = c_1 \cdot s_1 + c_2 \cdot s_2 + ...$) of variables $s_i$ of the so called solution $\vec{s}$ vector.


%%%%%%%%%%%%%%%%%%%%%%%%%%%%%%%%
\paragraph{R1CS satisfiability.}
	For all triplets ($\vec{a}, \vec{b}, \vec{c}$) of vectors in the R1CS, the solution vector $\vec{s}$ must satisfy 
$\brkt{\vec{a}, \vec{s}} \cdot \brkt{\vec{b}, \vec{s}} - \brkt{\vec{c}, \vec{s}} = 0$, 
where $\brkt{\cdot, \cdot}$ denotes the dot product of two vectors.
	The first element of $\vec{s}$ is fixed to the constant 1 (instead of a variable), to enable encoding constants in the constraints.
	The remaining elements represent several kinds of variables:
	\begin{itemize}
	\item \textbf{Witness variables:} 
	known only to the prover; represent external inputs to the constraint system --- the witness of the ZK proof system.
	\item \textbf{Internal variables}: 
	known only to the prover; internal to the constraint system (represent the inputs and outputs of multiplication gates);
	\item \textbf{Instance variables:} known by both prover and verifier.
	\end{itemize}

	A R1CS does not produce an output from an input (as for example a circuit does), but can be used to 
verify the correctness of a computation (e.g., performed by circuits with logic and/or arithmetic gates).
	The R1CS checks that the output variables (commonly known by both prover and verifier) are 
consistent with all other variables (possibly known only by the prover) in the solution vector.
	R1CS is only an intermediate representation, since the actual use in a ZKP system requires 
subsequent formulations (e.g., into a QAP) to enable verification without revealing the secret variables.


	A R1CS can be used to represent a Boolean circuit satisfiability 
problem and also to verify computations in arithmetic circuits.
	It is sufficient to observe that arbitrary circuits can be represented 
using multiplication and linear combination of polynomials, and these 
in turn correspond to R1CS constraints. For example:

\begin{itemize}

\item \textbf{Boolean circuits operations:}

	\begin{itemize}

	\item \textbf{\NOT\ operation:}
	If $x$ is a Boolean variable, then $1-x$ is the negation of $x$.
	Put differently, if $x$ is 0 or 1, then $1-x$ is respectively 1 or 1.

	\item \textbf{\AND\ operation:} 
		can be implemented as $(A) \times (B)$

	\item \textbf{\XOR\ operation ($c = a~\XOR~b$):}
		can be implemented as $(2 \cdot a ) \times (b) = (a + b - c)$, 
	or equivalently as $c = a + b - (a~\AND~b) \ast 2$
	\end{itemize}


\item \textbf{Arithmetic circuit operations:} 

	\begin{itemize}
	\item Multiplication gates are directly represented as equations of the form $a \ast b = c$.
	\item Linear constraints are used to keep track of inputs and outputs across these gates, 
and to represent addition and multiplication-by-constants.
	\end{itemize}

\end{itemize}


%%%%%%%%%%%%%%%%%%%%%%%%%%%%%%%%%%%%%%%%%%%%%%%%%%%%%%%%%%%%%%%%
\subsection{Types of relations}
\label{security:spec-statements-ZK:types-of-relations}

\revblock[rev:chap-security:circuits-vs-R1cs:new-subsection:types-of-relations]{\ref{it:circuits-vs-R1CS:new-subsections}}
 
\textbf{Special purpose relations:} Circuit satisfiability is 
a complete problem within the non-deterministic polynomial (NP) class,
i.e., it is NP-complete,
but a relation does not have to be that. 
	Examples of statements that appear in cryptographic usage include that a committed value falls in a certain range $[A;B]$ or belongs to a set $S$, that a ciphertext has plaintext 0 or that two ciphertexts encrypt the same value, that the prover has a secret key associated with a set of public verification keys for a signature scheme, etc.


\textbf{Setup-dependent relations:} 
Sometimes it is convenient to let the relation $R$ take an additional input setup$_R$, i.e., let the relation contain triples $(\textmd{setup}_R, x, w)$. 
	The input setup$_R$ can be used to specify persistent information, e.g., 
for arithmetic circuit satisfiability maybe the same finite field and circuit is used many times, 
so we let $\textmd{setup}_R = (F, C)$ and $x = (v_1,...,v_{n_x})$.
	The input setup$_R$ can also be used to capture trusted input the relation does not check, 
e.g., a trusted Rivest--Shamir--Adleman (RSA) modulus.

