\section{Specifying Statements for ZK}
\label{security:spec-statements-ZK}

 
	This document considers types of statements defined by a relation $R$ between instances $x$ and witnesses $w$.
	The relation $R$ specifies which pairs $(x,w)$ are considered related to each other, and which are not related to each other.
	The relation defines a matching language $L$ consisting of instances $x$ that have a witness $w$ in $R$.
	A statement is a claim $x \in L$, which can be true or false.
 
	The relation $R$ can for instance be specified as a program (e.g. in C or Java), which given inputs $x$ and $w$ decides to accept, meaning $(x,w) \in R$, or reject, meaning $w$ is not a witness to $x \in L$. 
	Examples of such specifications of the relation are detailed in the \hyperref[chap:apps]{Applications track}.
	In the academic literature, relations are often specified either as 
random access memory\luised{Introduced the extended expression of the acronym} (RAM)
programs or through %boolean 
Boolean
and arithmetic circuits, which we describe below.


\textbf{Circuits:} A circuit is a directed acyclic graph (DAG) comprised of nodes and labels for nodes, which satisfy the following constraints:
\begin{itemize}
\item Nodes with in-degree 0 are referred to as the \textbf{input nodes}\luissug{Consider adjusting all items to become of the form \textbf{<new term>}: <description>.} and are labeled with some constant (e.g., $0, 1, \ldots$) or with input variable names (e.g., $v_1, v_2, \ldots$)
\item There is a single node with out-degree 0 that is referred to as the \textbf{output node}.
\item Internal nodes\luised{``will be'' --> ``are'', consistent with the text in the first bullet} %will be 
are 
referred to as \textbf{gate nodes} and %will 
describe a computation %to be 
performed at the node.\loosen
\end{itemize}


%%%LB: Consider moving the `parametes' up, because it defines n_x and n_w, which are used in the subsequent bullets
\underline{Parameters.} Depending on the application, various parameters may be important, for instance the number of gates in the circuit, the number of instance variables $n_x$, the number of witness variables $n_w$, the circuit depth, or the circuit width.
 
\underline{Boolean Circuit satisfiability.} 
	The relation $R$ has instances of the form $x = (C, v_1, \ldots, v_{n_x})$ and witnesses $w = (w_1,...,w_{n_w})$. 
	For $(x,w)$ to be in the relation, $C$ must be a circuit with fan-in 2 gate nodes 
that are labeled with Boolean operations, e.g., \XOR\ or \AND, $v_1,...,v_{n_x}$ must specify 
truth values for some of the input nodes, and $w_1,...,w_{n_w}$ must specify 
truth values for the remaining input variables, such that when evaluating the circuit the output node becomes 1 (true).
 
\underline{Arithmetic Circuit satisfiability.} 
	The relation has instances of the form $x = (F, C, v_1,...,v_{n_x})$ and witnesses $w = (w_1,...,w_{n_w})$.
	For $(x,w)$ to be in the relation, $F$ must be a finite field (e.g., integers modulo a prime $p$), $C$ must be a circuit with gate nodes that are labeled with field operations, i.e., addition or multiplication, $v_1,...,v_{n_x}$ must specify field elements for some of the input nodes, and $w_1,...,w_{n_w}$ must specify field elements for the remaining input variables, such that when evaluating the circuit the output node becomes 1.
 

 
\textbf{Special purpose relations:} Circuit satisfiability is 
a complete problem within the non-deterministic polynomial\luised{introduced the extended form of NP} (NP) class,
i.e., it it NP-complete,
but a relation does not have to be that. 
	Examples of statements that appear in cryptographic usage include that a committed value falls in a certain range $[A;B]$ or belongs to a set $S$, that a ciphertext has plaintext 0 or that two ciphertexts encrypt the same value, that the prover has a secret key associated with a set of public verification keys for a signature scheme, etc.
 
\textbf{Setup-dependent relations:} 
Sometimes it is convenient to let the relation $R$ take an additional input setup$_R$, i.e., let the relation contain triples $(\textmd{setup}_R, x, w)$. 
	The input setup$_R$ can be used to specify persistent information, e.g., 
for arithmetic circuit satisfiability maybe the same finite field and circuit is used many times, 
so we let $\textmd{setup}_R = (F, C)$ and $x = (v_1,...,v_{n_x})$.
	The input setup$_R$ can also be used to capture trusted input the relation does not check, 
e.g., a trusted Rivest--Shamir--Adleman\luised{Introduced the extended form of the acronym.} (RSA)
modulus.

