\section{Introduction}
\label{security:intro}


%%%%%%%%%%%%%%%%%%%%%%%%%%%%%%%%%%%%%%%%%%%%%%%%
\subsection{What is a zero-knowledge proof?}
\label{security:intro:what-is-a-ZK}

	A zero-knowledge proof makes it possible to prove a statement is true while preserving confidentiality of secret information.
	There are numerous uses of zero-knowledge proofs. 
\reftab{tab:basic-examples-what-is-a-ZK}\luised{Changed from ``, the table below gives a few examples'' to ``Table 1.1 gives three examples''} %, the table below gives a few examples 
gives three example\luissug{Consider saying ``example scenarios'', since the table calls it ``scenarios''} where proving claims about confidential data can be useful.
\loosen


%\begin{table}[!h]
%\caption{Basic example scenarios for ZK proofs}
\begin{center}
\def\tempTabTitle{Basic example scenarios for ZK proofs}
\mytabcap{\tempTabTitle}{\tempTabTitle\luised{Introduced table caption and table number in this and in all other tables. In the first cell, introduced the qualifier "elements" to differentiate from "scenarios"}~~~\luissug{To enable adding new examples, consider transposing the table (i.e., one example scenario per row).}}
\label{tab:basic-examples-what-is-a-ZK}

\begin{edtable}{tabular}{|l||l||l||l|}   %\begin{tabular}{|l||l|l|l|}
\hline \rowcolor{colorRowHead}\diagbox{\small \bfseries \makebox[3em]{\hspace{1em}Elements}}{\small \bfseries \makebox[0pt]{\hspace*{-3.5em}Scenarios}}
		& \bfseries \subtab[l]{1. Legal age\\\hphantom{1. }for purchase}
		& \bfseries \subtab[l]{2. Hedge fund solvency} 
		& \bfseries \subtab[l]{3. Asset transfer} \\
\hline
\hline \bfseries Statement
		& I am an adult 
		& We are not bankrupt 
		& \subtab[l]{I own this asset}\\
\hline \bfseries \subtab[l]{Confidential\\information}
		& \subtab[l]{Exact age and\\personal data} 
		& Composition of portfolio 
		& Past transactions\\
%%%\hline \bfseries \subtab[l]{Supporting\\instance}
%%%		& \subtab[l]{Driver's license\\smartcard chip} 
%%%		& Encrypted bank records
%%%		& Blockchain block\\
\hline 
\end{edtable}\vspace{1em}
\end{center}

A zero-knowledge proof system is a specification of how a prover and verifier 
can interact for the prover to convince the verifier that the statement is true.
\luissug{Comment and suggestion.\textCR
    Contrary to the chess example in section 2.4, where there is no commitment as a starting point,
for each of the three examples in Table 1.1 there must be some substract (an ``instance'') 
to support the statement with respect to the confidential info.
	\textCR\textCR
	Without the ``instance'', a ZK proof in these cases seems like magic: 
how can I prove that the license place of my car starts with the letter ``C'' 
if there is no corresponding substract (e.g., a picture of the car, or a 
car-registration record? (do you even known whether or not I have a car?).
	\textCR\textCR
	Suggestion: Consider providing some intuition about the need for a supporting ``instance''.
Depending on where the intuition is added, a corresponding row ``Supporting instance'' could 
possibly be added to \reftab{tab:basic-examples-what-is-a-ZK}, e.g., with values 
``Driver's license smartcard chip''; ``Encrypted bank records''; ``Blockchain block''. 
	\textCR\textCR
	More generally, a reflection about how to position this may come from considering how to distinguish ZKPs of membership from ZKPoKs.
	\textCR\textCR
	Might the chessboard example (described in \refsec{apps:gadgets-within-predicates}) be
interesting to put in this table? It implicitly contains the supporting instance --- everything 
is about the chessboard and chess rules that the verifier already known.}
    The proof system must be complete, sound and zero-knowledge.
\begin{itemize}
\item \textbf{Complete:} If the statement is true and both prover and verifier follow the protocol; the verifier will accept.
\item \textbf{Sound:} If the statement is false, and the verifier follows the protocol; the verifier will not be convinced.
\item \textbf{Zero-knowledge:} If the statement is true and the prover follows the protocol; the verifier will not learn any confidential information from the interaction with the prover but the fact the statement is true.
\end{itemize}



%%%%%%%%%%%%%%%%%%%%%%%%%%%%%%%%%%%%%%%%%%%%%%%%
\subsection[Requirements for a ZK proof system specification]{Requirements for a zero-knowledge proof system specification}
\label{security:intro:requirements-ZK}

A full proof system specification MUST include:
\begin{enumerate}
\item Precise specification of the type of statements the proof system is designed to handle
\item Construction including algorithms used by the prover and verifier
\item If applicable, description of setup the prover and verifier use
\item Precise definitions of security the proof system is intended to provide
\item A security analysis that proves the zero-knowledge proof system satisfies the security definitions and a full list of any unproven assumptions that underpin security
\end{enumerate}

Efficiency claims about a zero-knowledge proof system should include all relevant performance parameters for the intended usage.
Efficiency claims must be reported fairly and accurately, and if a comparison is made to other zero-knowledge proof systems a best effort must be made to compare apples to apples.

The remainder of the document will outline common approaches to specifying 
a %LB
zero-knowledge proof system, outline some construction paradigms, and give guidelines for how to present efficiency claims.
\loosen

