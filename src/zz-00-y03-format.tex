

%%% Related to package titlesec
%%%\titlespacing{<command>}{<left>}{<before-sep>}{<after-sep>}[<right>]
%\titlespacing*{\chapter}{0pt}{-25pt}{0pt} %%-40pt
\titlespacing*{\chapter}{0pt}{-25pt}{2em} %%-40pt
\titleformat{\chapter}[display]{\normalfont\huge\centering}{Chapter \thechapter.\enskip}{20pt}{\Huge}
\titleformat{\chapter}[hang] 
{\normalfont\huge\bfseries}{\chaptertitlename\ \thechapter.}{.5em}{} 
\titlespacing{\subsubsection}{0pt}{1em}{0em}


%%% \presection: prints a centered and unnumbered section header, and adds the header to the table of contents
% 2nd optional argument has default value `section'
% e.g., \presection[short-section-name]{my-very-long-section-name}
% e.g., \presection[short-subsection-name][subsection]{my-very-long-subsection-name}
%\newcommandtwoopt{\presection}[3][][section]{\phantomsection\section*{\centering #3}\addcontentsline{toc}{#2}{\ifEmptyElse{#2}{#3}{#2}}}

\newcommandtwoopt{\presection}[3][][section]{\phantomsection\section*{\centering #3}\ifEmptyElse{#1}{\addcontentsline{toc}{#2}{#3}}{\addcontentsline{toc}{#2}{#1}}}

%\newcommand{\presection}[2][section]{\phantomsection\section*{\centering #2}\addcontentsline{toc}{#1}{#2}}
%\newcommand{\presection}[2][section]{\phantomsection\section*{\centering #2}\addcontentsline{toc}{#1}{#2}}


\renewcommand{\headrulewidth}{0pt}


%%%%%%%%%%%% Formatting
\theorembodyfont{\rm}\newtheorem{remark}{Remark} %needs package theorem
\setcounter{tocdepth}{3} % depth at which to show contents in the table of contents % needs package tocloft
\renewcommand\cftchapafterpnum{\vskip0pt} %spacing between items in toc
\renewcommand\cftsecafterpnum{\vskip0pt} %spacing between items in toc
\setlist[itemize]{itemsep=.5ex, topsep=0pt}
\setlist[enumerate]{itemsep=.5ex, topsep=0pt}
\setlength{\parskip}{.75em}
\setlength{\parindent}{0pt}

\renewcommand{\chaptermark}[1]{\markboth{#1}{}}
\renewcommand{\sectionmark}[1]{\markright{\thesection\ #1}}


\hypersetup{
	linktocpage,
	bookmarksnumbered,
	bookmarksopen,
	bookmarksdepth=3,
	bookmarksopenlevel=1,
	pdfdisplaydoctitle=true,
	pdfstartview={FitH},    % fits the width of the page to the window
	breaklinks=true,
	colorlinks=true,
	allcolors=darkblue,
}
\urlstyle{rm}



%%%%%%%%%%%% Related to package fancyhdr
%use \ifoddeven to define different headers between odd and even pages

\setlength{\headheight}{13.6pt}

\fancypagestyle{mainmatter}{\fancyhead{}
		\lhead{\ifoddeven{\nouppercase{\leftmark}}{}}
		\rhead{\ifoddeven{}{Section \nouppercase{\rightmark}}}
		\cfoot{\thepage}
}

\fancypagestyle{references}{\fancyhead{}
		\lhead{\ifoddeven{References}{}}
		\rhead{\ifoddeven{}{References}}
		\cfoot{\thepage}}

\fancypagestyle{appendix}{\fancyhead{}
		\lhead{\ifoddeven{\nouppercase{\leftmark}}{}}
		\rhead{\ifoddeven{}{Section \nouppercase{\rightmark}}}
		\cfoot{\thepage}
}

\def\intentiallyblank{\smash{\raisebox{-3.5em}{\scalebox{1.5}{\color[rgb]{0.85,0.85,0.85}Page intentionally blank}}}}
\fancypagestyle{intentionallyblank}{\fancyhead{}\chead{\intentiallyblank}\cfoot{}}


\fancypagestyle{blankpagewithnumber}{\fancyhead{}\chead{\intentiallyblank}\cfoot{\thepage}} %to avoid

\newcommand\cleartooddpage{\clearpage\ifodd\value{page}\else
	%avoid the \rightmark of the `mainmatter' page style, but still include a page number
	\null\thispagestyle{blankpagewithnumber}\clearpage\fi}

\fancypagestyle{empty}{\fancyhead{}\cfoot{}}


%%% Same font size as section header ?
\renewcommand\cfttoctitlefont{\bfseries\Large}
\renewcommand\cftloftitlefont{\bfseries\Large}
\renewcommand\cftlottitlefont{\bfseries\Large}

\makeatletter
\@ifpackageloaded{babel}{%
	%if using babel
	\addto\captionsenglish{% Replace "english" with the language you use %is using package `babel', must specify the language
		\renewcommand{\contentsname}{\phantomsection\pdfbookmark[2]{Table of Contents}{pdfbkm:toc}\label{toc:toc}Table of Contents}
		\renewcommand{\bibname}{\pdfbookmark[1]{References}{pdfbkm:refs}References}
		}%
		%%% \renewcommand{\refname}
	}%
	{%
		\renewcommand{\contentsname}{\phantomsection\pdfbookmark[2]{Table of Contents}{pdfbkm:toc}\label{toc:toc}Table of Contents}%
		\renewcommand{\bibname}{\pdfbookmark[1]{References}{pdfbkm:refs}References}
	}
\makeatother


\DeclareListWrapperFormat{publisher}{Pub.\ by \mkbibemph{#1}}
%\DeclareListWrapperFormat{organization}{\mkbibemph{#1}}


%%%\newcommand{\myciton}{cited on}
%%%\newcommand{\mybibstring}{}
%%%\DeclareListFormat{list:citon}{%
%%%    \ifnumless{\value{listcount}}{\value{liststop}}
%%%    {\ifdefstring{\titlepagename}{#1}{\myciton\ppspace\listbreak}{}}
%%%    {\ifdefstring{\titlepagename}{#1}{\myciton}{\mybibstring}\ppspace}}
%%%\renewbibmacro*{pageref}{\iflistundef{pageref}{}
%%%    {\printtext[parens]{%
%%%       \ifnumgreater{\value{pageref}}{1}
%%%         {\renewcommand{\mybibstring}{\bibstring{backrefpages}}}
%%%         {\renewcommand{\mybibstring}{\bibstring{backrefpage}}}%
%%%       \printlist[list:citon]{pageref}
%%%       \printlist[pageref][-\value{listtotal}]{pageref}}}}
			

%%% https://tex.stackexchange.com/questions/149009/biblatex-pagebackref-reference-in-the-flush-right-margin
\renewbibmacro*{pageref}{%
   \iflistundef{pageref}
     {\renewcommand{\finentrypunct}{\addperiod}}
     {\renewcommand{\finentrypunct}{\addspace}%
      \printtext{\addperiod\hfill\rlap{\hskip15pt\colorbox{blue!5}{\scriptsize\printlist[pageref][-\value{listtotal}]{pageref}}}}}}


\newcommand\twodigits[1]{\ifnum#1<10 0#1\else #1\fi}
\def\todayext{\ifcase\month\or
  January\or February\or March\or April\or May\or June\or
  July\or August\or September\or October\or November\or December\fi
  \space\twodigits{\number\day}, \number\year}