\chapter{Version history}
\label{app:version-history}


\revblock[rev:editorial:add-version-history]{\ref{it:editorial:add-version-history}}
The development of the ZKProof Community reference can be tracked across a sequence of main versions.
Here is a summarized description of their sequence:

\begin{itemize}\setlength{\itemsep}{1em}

\item \textbf{Version 0 [2018-08-01]: Baseline documents.}
	The proceedings of the 1st ZKProof Workshop (May 2018), 
with contributions settled by 2018-08-01 and available 
at \href{https://zkproof.org/documents}{ZKProof.org},
along with the \hyperref[sec:prelim:charter]{ZKProof Charter}, 
constitute the starting point of the ZKProof Community reference.
	Each of the three Workshop tracks --- security, applications, implementation --- 
lead to a corresponding proceedings document, 
named ``ZKProof Standards \emphbrkt{track name} Track Proceedings''.
	The ZKProof charter is also part of the baseline documents.


\item \textbf{Version 0.1 [2019-04-11]: LaTeX/PDF compilation.}
	Upon the ZKProof organization team requested feedback from the NIST-PEC team, the content in the 
several proceedings was ported to LaTeX code and compiled into a single PDF document entitled 
``ZKProof Community Reference'' (version 0.1) for presentation and discussion at the 2nd ZKProof workshop.
	The version includes editorial adjustments for consistent style and easier indexation. 


\item \textbf{Version 0.2 [2019-12-31]: Consolidated draft.}
	The process of consolidating the draft community reference document started at the 
2nd ZKProof workshop (April 2019), where an editorial process was introduced and 
several ``breakout sessions'' were held for discussion on focused topics, including
the ``NIST comments on the initial ZKProof documentation''.
	The discussions yielded suggestions of topics to develop and incorporate in a new version of the document.
	Several concrete items of ``proposed contributions'' were then defined as GitHub issues,
and the subsequently submitted contributions provided several content improvements, such as:
	distinguish ZKPs of knowledge vs.\ of membership;
	recommend security parameters for benchmarks;
	clarify some terminology related to ZKP systems (e.g., statements, CRS, R1CS);
	discuss interactivity vs.\ non-interactivity, and transferability vs.\ deniability;
	clarify the scope of use-cases and applications; 
	update the ``gadgets'' table; add new references.
	The new version also includes numerous editorial improvements towards a consolidated document, 
namely a substantially reformulated frontmatter with several new sections
(abstract, open to contributions, change log, acknowledgments, intellectual property, executive summary),
a reorganized structure with a new chapter (still to be completed) on construction paradigms.
	The changes are tracked in a ``diff'' version of the document.
\end{itemize}


%%%%%%%%%%%%%%%%%%%%%%%%%%%%%%%%%%%%%%%%%%%%%%%%%%%%%%%%%%%%
\paragraph{External resources.}%\pdfbookmark[2]{External resources}{pdfbkm:ext-res}
\label{par:prelim:change-log:external-resources}

Additional documentation covering the history of development of this 
community reference can be found in the following online resources:
\begin{itemize}[topsep=0pt,itemsep=1ex]
	\item ZKProof GitHub repository: \myurl{https://github.com/zkpstandard/}
	\item ZKProof documentation: \sloppy\mbox{\myurl{https://zkproof.org/documents.html}}
	\item ZKProof Forum: \myurl{https://community.zkproof.org/}
\end{itemize}
