%%%%%%%%%%%%%%%%%%%%%%%%%%%%%%%%%%%%%%%%%%%%%%%%%%%%%%%%%%%%
%%%%%%%%%%%%%%%%%%%%%%%%%%%%%%%%%%%%%%%%%%%%%%%%%%%%%%%%%%%%
\section{Introduction and Motivation}
\label{apps:intro}
 
In this track we aim to overview existing techniques for building ZKP based systems, including designing the protocols to meet the best-practice security requirements. One can distinguish between high-level and low-level applications, where the former are the protocols designed for specific use-cases and the latter are the underlying operations needed to define a ZK predicate. We call gadgets the sub-circuits used to build the actual constraint system needed for a use-case. In some cases, a gadget can be interpreted as a security requirement (e.g.: using the commitment verification gadget is equivalent to ensuring the privacy of underlying data). 

As we will see, the protocols can be abstracted and generalized to admit several use-cases; similarly, there exist compilers that will generate the necessary gadgets from commonly used programming languages. Creating the constraint systems is a fundamental part of the applications of ZKP, which is the reason why there is a large variety of front-ends available.

In this document, we present three use-cases and a set of useful gadgets to be used within the predicate of each of the three use-cases: identity framework, asset transfer and regulation compliance.
%%% Commented out, since it is addressed by newly added paragraph
%%%\pdfcomment[author={Eran Tromer}]{Missing coverage of recursive composition and Proof-Carrying Data as an important high-level tool (in existing prototypes and emerging real applications). There's a couple of brief mentions in the Implementation track, but those don't cover the usefulness in applications.}


%%% NOTE LB: Introduced minor editorial changes (all can be reconsidered by the contributor)
\revblock[rev:app:intro:motivate]{\ref{it:improve-motivation-apps}}
	The design of ZKPs is subject to the tradeoff between functionality and performance. 
	Users would like to have powerful ZKPs, in the sense that the system permits constructing %to construct 
proofs for any predicate, which %what 
leads to the necessity of universal ZKPs. 
	On the other hand, users would like to have efficient constructions. 
	According to Table~\ref{tab:apps:APIS-and-interfaces-by-univ-and-preproc}, it is possible to classify ZKPs as:
(i) universal or non-universal;
(ii) scalable or non-scalable; and 
(iii) preprocessing or non-preprocessing. 
	Item (i) is related to the functionality of the underlying %subjacent 
ZKP, while items (ii) and (iii) are related to performance. 
	The utilization of zk-SNARKs allows %to have 
universal ZKPs with very efficient verifiers. 
	However, most proposals %%% LB: ``most'' is very contextual to a moment in time; consider revising, e.g., ``many'' or ``some''
depend upon an expensive preprocessing, which makes such systems hard to scale for some use-cases. 
	A technique called \textit{Proof-Carrying Data} (PCD), originally proposed in Ref.~\cite{2010:ICS:proof-carrying-data},\revblock[rev:app:intro:new-refs]{\ref{it:refs-in-chapter-apps}} 
allows obtaining %to obtain 
\textit{recursive composition} for existing ZKPs in a modular way. %, which
	This means that zk-SNARKs can be used as a building block to construct scalable and non-preprocessing solutions.
	%thus obtaining 
	The result is not only an efficient verifier, as in zk-SNARKs, but also a prover whose consumption of computational resources is efficient, in particular with respect to memory requirements, as described in Refs.~\cite{2017:Alg:Scalable-Zero-Knowledge-Via-Cycles-of-Elliptic-Curves} and~\cite{2013:Recursive-Composition-and-Bootstrapping-for-SNARKS-and-Proof-carrying-Data}.
\loosen


%%%%%%%%%%%%%%%%%%%%%%%%%%%%%%
\paragraph{What this document is NOT about:}
\begin{itemize}
 \item A unique explanation of how to build ZKP applications
 \item An exhaustive list of the security requirements needed to build a ZKP system
 \item A comparison of front-end tools
 \item A show of preference for some use-cases or others
\end{itemize}

