%%%%%%%%%%%%%%%%%%%%%%%%%%%%%%%%%%%%%%%%%%%%%%%%%%%%%%%%%%%%
\presection{Change log}
\label{sec:prelim:change-log}


\revblock[rev:editorial:add-change-log]{\ref{it:editorial:add-change-log}}
The development of the ZKProof Community reference can be tracked across a sequence of main versions.
Here is a summarized description of the sequence of main versions:

\begin{itemize}\setlength{\itemsep}{1em}

\item \textbf{2018-08-01: Version 0 --- Baseline documents.}
	The proceedings of the 1st ZKProof Workshop (May 2018), 
with contributions settled by 208-08-01 and available 
at \href{https://zkproof.org/documents}{ZKProof.org},
constitute the starting point of the ZKProof Community reference.
	Each of the three Workshop tracks --- security, applications, implementation --- 
lead to a corresponding proceedings document, 
named ``ZKProof Standards \emphbrkt{track name} Track Proceedings''.


\item \textbf{2019-04-11: Version 0.1 --- LaTeX/PDF compilation.}
	Upon the ZKProof organization requested for feedback by the NIST-PEC team, the content in the 
several proceedings (version 0) was ported to LaTeX code and compiled into a single PDF document 
entitled ``ZKProof Community Reference'' for presentation and discussion at the 2nd ZKProof workshop.
	The version includes editorial adjustments for consistent style and easier indexation of content.


\item \textbf{2019-12-\red{xx}: Version 0.2 --- Consolidated draft.}
	The consolidation of the draft community reference followed a new editorial process initiated at the 2nd ZKProof workshop.
	The workshop held several ``breakout sessions'' for discussion on focused topics.
	Some of those sessions yielded suggestions for new/adjusted content for a new version, 
and subsequently several concrete items were defined as GitHub issues.
	The subsequently submitted contributions relate to several topics:
distinguish ZKPs of Knowledge vs.\ of membership, recommend security parameters for benchmarks, 
clarify some terminology related to ZKP systems (e.g., statements, CRS, R1CS);
discuss transferability and deniability; clarify the scope of use-cases and applications; 
update the ``gadgets'' table; add some new references; 
include a new chapter on construction paradigms, including discussion on interactivity vs.\ non-interactivity.
	The new version also includes numerous editorial improvements towards a consolidated document, 
namely a substantially reformulated frontmatter with several new sections: 
abstract, open to contributions, change-log, acknowledgments, intellectual property, executive summary.
	The changes are tracked in an available ``diff'' version.
\end{itemize}


%%%%%%%%%%%%%%%%%%%%%%%%%%%%%%%%%%%%%%%%%%%%%%%%%%%%%%%%%%%%
\paragraph{External resources.}%\pdfbookmark[2]{External resources}{pdfbkm:ext-res}
\label{par:prelim:change-log:external-resources}

Additional documentation covering the history of development of this 
community reference can be found in the following online resources:
\begin{itemize}[topsep=0pt,itemsep=1ex]
	\item ZKProof GitHub repository: \myurl{https://github.com/zkpstandard/}
	\item ZKProof documentation: \sloppy\mbox{\myurl{https://zkproof.org/documents.html}}
	\item ZKProof Forum: \myurl{https://community.zkproof.org/}
\end{itemize}
