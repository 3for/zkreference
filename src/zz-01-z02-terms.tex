\section{Terminology}
\label{security:terminology}


\luissug{Consider defining upfront the symbols for prover (\prov) and verifier (\veri), 
since the symbols will often be used as subscripts and/or as abbreviated forms.
\textCR\textCR
For example: ``A zero-knowledge proof exists in a context of two parties with different roles --- 
the prover produces the proof; the verifier verifies the proof.
Notation: \prov{} (prover); \veri{} (verifier).
Some generalizations consider the existence of several verifiers.''}
%%%%%
%%%%%
{\bfseries \hypertarget{def:instance}{Instance}:} Public input\luissug{Change ``public input'' to ``common input''.
Rationale: ZK proofs make sense even in a context private to prover-and-verifier, where the input made available to the verifier is not supposed to be disclosed publicly, e.g., to avoid linkability in a broader context;
while the use of ``public'' may be usual (e.g., present when mentioning ``private|public'' key systems, 
where ``private'' means ``secret'', and where ``public'' may mean ``private but shareable''), 
a more cautious use (e.g., ``common'') can be more informative about the actual requirement --- 
that it be known by both prover and verifier.
} 
that is known to both prover and verifier.
%LB: Consider changing to ``Scientific articles sometimes'' or ``Certain scientific articles use'' or ``Some works use''
Sometimes scientific articles use “instance” and “statement” interchangeably, but we %will 
distinguish between the two. 
Notation: $x$.
\loosen
 
{\bfseries  \hypertarget{def:witness}{Witness}:} 
Private input to the prover. Others may or may not know something about the witness. 
Notation: $w$.
 
{\bfseries \hypertarget{def:relation}{Relation}:}\luissug{Consider organizing the set of definitions in two or three classes: 
\textCR: a) participants (prover and verifier);
\textCR: b) elements instantiated in each proof execution (instance, witness, statement, setup);
\textCR: c) elements defining a proof system (language, relation) that define the proof system;} 
	Specification of relationship between instances and witness.
	A relation can be viewed as a set of permissible pairs (instance, witness). 
Notation: $R$.
 
{\bfseries \hypertarget{def:language}{Language}:} 
Set of instances that appear as a permissible pair in $R$. 
Notation: $L$.
 
{\bfseries \hypertarget{def:statement}{Statement}:} 
Defined by instance and relation. Claims the instance has a witness in the relation (which is either true or false). 
Notation: $x \in L$.

{\bfseries Security parameter:} 
Positive integer indicating the desired security level (e.g. 128 or 256) where higher security parameter means greater security.
In most constructions, distinction is made between computational security parameter and statistical security parameter. 
Notation: $k$ (computational) or $s$ (statistical).
 
{\bfseries  \hypertarget{def:setup}{Setup}:} Input to e.g. prover and verifier\luissug{Include ``CRS'' as a sub-item of the ``setup''; differentiate private setup from common setup; make a concrete definition, rather than one based on example "e.g.";\textCR\textCR
Concrete suggestion: "Setup: The inputs given to the prover and to the verifier, independent from the instance x and the witness w. The setup of each party can be decomposed into a private component (``\privsetP'' or ``\privsetV'', respectively not known to the other party) and a common component ``CommonSetup = CRS'' (known by both parties). Notation: \setP\ = (\privsetP, CRS) and : \setV\ = (\privsetV, CRS), where CRS denotes a ``\textbf{common reference string}'' (required by some zero-knowledge proof systems)."
\textCR\textCR
Another option could be to define the common setup to include x, and the private setup of P to include w, but that does not seem to be the notation option used in the subsequent sections.
}

{\bfseries Common reference string:} Some zero-knowledge systems require common public input, e.g., CRS = \setP\ = \setV.\luissug{Comment and suggestion: As is, the text may convey (there is an ``e.g.'' but it's easy to miss it) 
	that the existence of a CRS implies that the full setup of \prov\ is equal to the full setup of \veri.
	Suggestion --- explicitly define CRS as the CommonSetup component of the setup, regardless of whether or not there is a PrivateSetup components. Then if useful provide two examples, one where a CRS is enough; another where there is for example a PKI with secret keys.
	\textCR The text, using ``common public'', also seems to convey that the CRS must be public.
	Suggestion: remove "public" and require only the ``common'' aspect.
	For example, being public vs. common-but-not-public may make the difference between a ZK proof being transferable or not.}



\reftab{tab:example-scenarios-zkps} exemplifies at a high level a differentiation between the \emph{statement}, the \emph{instance} and the \emph{witness} elements for the initial examples mentioned in \refsec{security:intro:what-is-a-ZK}.


\begin{table}[H]\centering
\mytabcap{Example scenarios for zero-knowledge proofs}{Example scenarios for zero-knowledge proofs}\label{tab:example-scenarios-zkps}\revblock[rev:ZKP:enhance-table-of-examples]{\ref{it:ZKP:enhance-table-of-examples}}

\vspace{1em}\small
\newcommand{\scaleTitle}[1]{\scalebox{.95}{#1}}
%\resizebox{\textwidth}{!}{%
\begin{tabular}{|l|l||l||l||l|}
\hline \rowcolor{colorRowHead}
			\bfseries \scalebox{.85}{\#}
		& \diagbox{\small \bfseries \makebox[2.25em]{\hspace{1em}\scalebox{.8}{Scenarios}}}{\small \bfseries \makebox[0pt]{\hspace*{-2.75em}\scalebox{.8}{Elements}}}
		& \subtab[l]{{\bfseries Statement}\\being proven}
		& \subtab[l]{{{\bfseries Instance}}\\used as substrate}
		& \subtab[l]{{\bfseries Witness}\\treated as confidential}\\
\hline
\hline \bfseries 1 
		& \bfseries \scaleTitle{\subtab[l]{Legal age for\\purchase}}
		& I am an adult 
		& \subtab[l]{Tamper-resistant\\identification chip}
		& \subtab[l]{Birthdate and personal\\data (signed by a CA)}\\
\hline \bfseries 2 
		&	\bfseries \scaleTitle{\subtab[l]{Hedge fund\\solvency}}
		& \subtab[l]{We are not bankrupt}
		& \subtab[l]{Encrypted \& certified\\bank records}
		& \subtab[l]{Portfolio data and\\decryption key}\\
\hline \bfseries 3
		&	\bfseries \scaleTitle{\subtab[l]{Asset\\transfer}}
		& I own this $<$asset$>$
		& \subtab[l]{A blockchain or\\other commitments}
		& \subtab[l]{\scalebox{1}{Sequence of transactions}\\\scalebox{1.0}{(and secret keys that}\\\scalebox{1.0}{establish ownership)}}\\
\hline \bfseries 4
		& \bfseries \scaleTitle{\subtab[l]{Chessboard\\configuration}}
		& \subtab[l]{This $<$configuration$>$\\can be reached}
		& \subtab[l]{(The rules of Chess)}
		& \subtab[l]{A sequence of valid\\chess moves}\\
\hline \bfseries 5
		& \bfseries \scaleTitle{\subtab[l]{Theorem\\validity}}
		& \subtab[l]{This $<$expression$>$\\is a theorem}  %(provable\\in $<n$ steps)
		& \subtab[l]{(A set of axioms,\\and the logical \\rules of inference)}   
		& \subtab[l]{A sequence of logical\\implications}\\
\hline 
\end{tabular}
%}%end of resizebox

\vspace{.25em}\begin{minipage}{1\textwidth}{\footnotesize Legend: CA = certification authority}\end{minipage}
\end{table}