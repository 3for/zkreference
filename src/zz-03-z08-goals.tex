%%%%%%%%%%%%%%%%%%%%%%%%%%%%%%%%%%%%%%%%%%%%%%%%%%%%%%%%%%%%
%%%%%%%%%%%%%%%%%%%%%%%%%%%%%%%%%%%%%%%%%%%%%%%%%%%%%%%%%%%%
\section{Future goals}
\label{implem:goals}


%%%%%%%%%%%%%%%%%%%%%%%%%%%%%%
\subsection{Interoperability}
\label{implem:goals:interoperability}

Many additional aspects of interoperability remain to be analyzed and supported by standards, to support additional ZK proof system backends as well as additional communication and reuse scenarios. Work has begun on multiple fronts both, and a dedicated public \href{https://groups.google.com/a/zkproof.org/forum/\#!forum/interoperability}{mailing list} is established.

\textbf{Additional forms of interoperability.}
As discussed in the Extended Constraint-System Interoperability section above, even within the R1CS realm, there are numerous additional needs beyond plain constraint systems and assignment representations. These affect security, functionality and ease of development and reuse.


\textbf{Additional relation styles.}
The R1CS-style constraint system has been given the most focus in the Implementation Track discussions in the first workshop, leading to a file format and an API specification suitable for it. It is an important goal to discuss other styles of constraint systems, which are used by other ZK proof systems and their corresponding backends. This includes arithmetic and Boolean circuits, variants thereof which can exploit regular/repeating elements, as well as arithmetic constraint satisfaction problems.


\textbf{Recursive composition.}
The technique of recursive composition of proofs, and its abstraction as Proof-Carrying Data (PCD) 
\cite{2010:ICS:proof-carrying-data,2014:crypto:Scalable-Zero-Knowledge-via-Cycles-of-Elliptic-Curves}, %[CT10][BCTV14]
can improve the performance and functionality of ZK proof systems in applications that deal with multi-stage computation or large amounts of data. This introduces additional objects and corresponding interoperability considerations. For example, PCD compliance predicates are constraint systems with additional conventions that determine their semantics, and for interoperability these conventions require precise specification.


\textbf{Benchmarks.}
We strive to create concrete reference benchmarks and reference platforms, to enable cross-paper milliseconds comparisons and competitions.

We seek to create an open competition with well-specified evaluation criteria, to evaluate different proof schemes in various well-defined scenarios.


%%%%%%%%%%%%%%%%%%%%%%%%%%%%%%
\subsection{Frontends and DSLs}
\label{implem:goals:frontends-and-DSLs}

We would like to expand the discussion on the areas of domain-specific languages, specifically in aspects of interoperability, correctness and efficiency (even enabling source-to-source optimisation).

The goal of Gadget Interoperability, in the Extended Constraint-System Interoperability section, is also pertinent to frontends.


%%%%%%%%%%%%%%%%%%%%%%%%%%%%%%
\subsection{Verification of implementations}
\label{implem:goals:verification-of-implementations}

We would to discuss the following subjects in future workshops, to assist in guiding towards best practices: formal verification, auditing, consistency tests, etc.
