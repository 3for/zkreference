%%% Luis B: ADD-ON TO INITIATE PART WITH TABLES OF NOTES, IN THE SAME DOCUMENT, 
%%% (CHANGED TO LANDSCAPE) TO CROSS-REFERENC FEEDBACK NOTES AND CORRESPONDING CHANGES
%%% It's appearance vs. not-appearing is controlled with the bit flag boolShowFeedbackNotes

%%% CHANGE PAGE FORMAT MID-DOCUMENT
\KOMAoption{paper}{landscape}%
\typearea{12}% sets new DIV ??
\recalctypearea
\newgeometry{margin=.65in,footskip=.25in}


\def\mytextfooter{Table of proposed/implemented contributions}
\fancypagestyle{lscapetablecontribs}{
	\fancyhf{}\lfoot{\mytextfooter}\rfoot{\thepage}
}

	%%%%%%%%%%%%%%%%%%%%%%%%%%%%%%%%%%%%%%%%%		
	%\clearpage\newpage
	\pagebreak
	\thispagestyle{lscapetablecontribs}%
	\pagestyle{lscapetablecontribs}%
	

	\def\tempTitle{Table of comments and contributions v0.1 \tops{$\rightarrow$}{\textrightarrow} 0.2}
	\phantomsection%\section*{\tempTitle}
	~\hfill\scalebox{2}{\textbf{\tempTitle}}\hfill~
	\pdfbookmark[0]{\tempTitle}{pdfbkm:table-of-comments}
	%%%\addcontentsline{toc}{chapter}{\tempTitle}

	\vspace{1em}
	The following pages describe contributions integrated in the process of upgrading the draft reference 
document from version 0.1 (dated 2019-04-11, available during the 2nd ZKProof Workshop) to version 0.2.


%%%%%%%%%%%%%%%%%%%%%%%%%%%%%%%%%%%%%%%%%%%%%%
\def\tmpSecTitle{Explanation of the tables of contributions}
\section*{\tmpSecTitle}\pdfbookmark[2]{\tmpSecTitle}{pdfbkm:explain-table-contributions}

Each table describes proposed contributions and corresponding changes/edits 
made to the baseline version 0.1, in order to achieve version 0.2.
Each table, indexed as D$x$ (where $x$ is an integer), corresponds to 
a \href{https://github.com/zkpstandard/zkreference/issues}{GitHub issue} 
(GI$y$, where $y$ is an integer) describing proposed contributions --- 
see \myurl{https://github.com/zkpstandard/zkreference/issues}.
	However, compared with GitHub, the description here can be adjusted and better detailed
for the purpose of enabling a better cross-referencing to the edits made in the document.
	Each table has a header as follows:

{
\vspace{-1em}
\renewcommand{\thecntIssue}{D$x$}
\renewcommand{\rowIssueName}{\emph{issue title}}
\beginlongtabIssue\hline\rowcolor{LLyellow}\headerForIssue\rowend\hline\end{longtable}
%\renewcommand{\thecntIssue}{D\arabic{cntIssue}}
}

From left to right, the columns represent:

\begin{itemize}

\item \bm{\#}:
	A consecutive positive integer, used to count all described items of contribution

\item \textbf{Item id} 
	An index of the contribution item, with a numbering subordinate to the table index. 
	For example, D1.3 would be the third item (row) of contribution (table) D1.
%	Each row shows D$x$.$z$ as the item id, where $z$ is an integer 
%and D.$x$ is the identification of the table.
	
\item \textbf{Location:} 
	A hint about the location (typically in the old document) of the edits.

\item \textbf{Contribution D$x$: \emph{issue title}}: 
	An identifier \textbf{D$x$} (with integer $x$) of the contribution 
\textbf{d}escription, and a title of the issue / contributions.

\item \textbf{Related}: Related references, 
such as references (GI$x$) to GitHub issues, 
and/or ids of other contribution items.

\item \textbf{Context and changes}: 
	Column to fill-in with contextual information about the proposed contribution,
as well as a high level description of the changes in the document.

\item \textbf{Edit id}:
	Index (or possibly several indices) of the edits (E$y$, with integer $y$) made
in the document.
	Across the document, changes will be marked in the right margin with this index,
so that the reader can hyperlink it directly to the description of the contribution,
i.e., to an explanation of why the change was made.
\looseness=-1

\end{itemize}

%%% LIST OF CONTRIBUTIONS
\clearpage
\phantomsection%%\addcontentsline{toc}{section}{\listcontributionname}
\pdfbookmark[1]{\listcontributionname}{pdfbkm:list-of-contributions}


{
\setlength{\cftbeforeloctitleskip}{2em}
\setlength{\cftsecindent}{0em}
\setlength{\cftsecnumwidth}{0em}
\renewcommand\cftloctitlefont{\bfseries\LARGE}
\addtocontents{loc}{\vspace*{-3em}}
\listofcontribution
}

%%%\newcommand{\itemlinerev}[2]{\item \ref{#1} --- #2 \hspace{.5em}\dotfill\hspace{.5em}\pageref{#1}}

