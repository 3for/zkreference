%%%%%% Description of relevant commands:
%%% \newIssue{label}{issue title}  % starts a new longtable (can break across pages) of comments
%%% \incItem[optional-label][optional-pdf-bookmark]  % starts a new row (`item') related to an issue
%%% \newcol                        % starts a new column --- see below a description of the columns
%%% \rowend                        % ends a row (`item')
%%% \myendIssue                    % ends a longtable associated to an `issue'

%%%For each new item (started with \incItem[...][...] and ended with \rowend), these are the columns:
	%[ the following initial columns are automatic: \#, ref]
	% Location   % before the first \newcol
	% Proposed contribution (\propContrib) and Contributors (\contributors)
	% Related
	% `Context' (\ccontext) and `Changed' (\Chan)
	% Edit id (References of the edits made along the document)


%%%%%%%%%%%%%%%%%%%%%%%%%%%%%%%%%%%%%%%%%%%%%%%%%%%%%%%%%%%%%%%%%%%%%%%%%%%%%%%%%%%%%%%%%%%%%%%%
%%%%%%%%%%%%%%%%%%%%%%%%%%%%%%%%%%%%%%%%%%%%%%%%%%%%%%%%%%%%%%%%%%%%%%%%%%%%%%%%%%%%%%%%%%%%%%%%
\def\tmpTitle{Structural changes by the editors}
\section*{\pdfbookmark[1]{\tmpTitle}{pdfbkm:structural-changes}\tmpTitle}
\label{sec:comments:structural-changes}

%%% Inherently related to the editorial development of the reference document

%%%%%%%%%%%%%%%%%%%%%%%%%%%%%%%%%%%%%%%%%%%%%%%%%%%%%%%%%%%%%%%%%%%%%%%%%%%%%%%%%%%%%%%%%%%%%%%%
\newIssue{issue:editorial-structural}{Implement editorial structural changes} %C16
%%%%%%%%%%%%%%%%%%%%%%%%
\incItem[it:editorial-structural][Structural changes]
All document
\newcol \ccontext\ Inherently related to the editorial development of the reference document. 
				\propContrib\ Implement editorial structural changes to the document (e.g., new chapters, 
				sections, subsections, etc.) , as useful based on the overall set of contributions.
\newcol \githubissue{16}, \ref{it:interactivity:intro}, \ref{it:transferability-vs-interactivity-elaborate}, \ref{it:deniability}
\newcol \contributors\ The editors (Daniel Benarroch, Luís Brandão, Eran Tromer)
				\Chan\ See items below.
\newcol 
\rowendL
%%%%%%%%%%%%%%%%%%%%%%%%
\incItem[it:editorial:cover-page][Cover page]
\hyperref[prelim:cover]{Cover}
\newcol 
\newcol \githubissue{16}
\newcol \Chan\ Update the version number; update the version date; add a link to find the latest version; add a ZKProof logo.
\newcol \ref{rev:update-cover-version-and-date}, \ref{rev:update-cover-link}
\rowendL
%%%%%%%%%%%%%%%%%%%%%%%%
\incItem[it:editorial:clarify-this-version][About this version]
Front matter
\newcol 
\newcol \githubissue{16}
\newcol \Chan\ Add a note ``\hyperref[sec:prelim:msg-editors]{about this version}'' clarifying the context of the current version; add a proposed citation format for this version.
\newcol \ref{rev:clarify-this-version}
\rowendL
%%%%%%%%%%%%%%%%%%%%%%%%
\incItem[it:editorial:propose-citation-format][Citation format]
Front matter
\newcol 
\newcol \githubissue{16}
\newcol \Chan\ Add a \hyperref[how-to-cite-this-version]{proposed citation format} for this version.
\newcol \ref{rev:propose-citation-format}
\rowendL
%%%%%%%%%%%%%%%%%%%%%%%%
\incItem[it:editorial:add-abstract][Abstract]
Front matter
\newcol 
\newcol \githubissue{16}
\newcol \Chan\ Add an abstract and a list of keywords
\newcol \ref{rev:add-abstract}
\rowendL
%%%%%%%%%%%%%%%%%%%%%%%%
\incItem[it:editorial:about-this-community-reference][About this community reference]
Front matter
\newcol \ccontext\ A significant portion of the text in this section 
is based on the ``Towards a reference document'' section of the 
``NIST comments on the initial ZKProof documentation'' (April 06, 2019).
\newcol \githubissue{16}
\newcol \Chan\ In the preamble of the document, add a section ``\hyperref[sec:prelim:about-this-community-reference]{About this community reference}'' providing context about the intended development process of the document.
\newcol \ref{rev:prelim:explain-context-of-community-reference}
\rowendL
%%%%%%%%%%%%%%%%%%%%%%%%
\incItem[it:editorial:charter][Charter]
Front matter
\newcol 
\newcol \githubissue{16}
\newcol \Chan\ Improve the placement and context of the ZKProof Charter within the document:
				\begin{itemize}
				\item Move the original ``\hyperref[sec:prelim:charter]{ZKProof Charter}'' to before the \hyperref[prelim:contents]{Table of Contents}, and frame it within a box (\ref{rev:charter:box-the-original}).
				\item Correct typo: ``standardardization'' $\rightarrow$ ``standardization'' (\ref{rev:charter:typo:standardardization}).
				\end{itemize}
\newcol \ref{rev:charter:box-the-original}, \ref{rev:charter:typo:standardardization}
\rowendL
%%%%%%%%%%%%%%%%%%%%%%%%
\incItem[it:editorial:charter-footnote-scope-cc-license][CC license]
Front matter
\newcol 
\newcol \githubissue{16}, \ref{issue:intellectual-property}
\newcol \Chan\ Add editorial footnote explaining that the scope of the creative commons license is widened to incorporate the community reference (\ref{rev:charter:widen-scope-creative-commons}).
\newcol \ref{rev:charter:widen-scope-creative-commons}
\rowendL
%%%%%%%%%%%%%%%%%%%%%%%%
\incItem[it:editorial:code-conduct][Intellectual property]
Front matter
\newcol 
\newcol \githubissue{16}, \ref{issue:intellectual-property}
\newcol \Chan\ Remove the ZKProof Code of Conducts (since it is tailored to events, rather than to documents).
\newcol 
\rowendL
%%%%%%%%%%%%%%%%%%%%%%%%
\incItem[it:editorial:toc-depth][Depth of table of contents]
Front matter
\newcol 
\newcol \githubissue{16}, \ref{rev:table-of-contents-depth-to-subsections}
\newcol \Chan\ Increase the depth of the table of contents to also show subsections
\newcol 
\rowendL
%%%%%%%%%%%%%%%%%%%%%%%%
\incItem[it:new-chapter-2][New chapter construction paradigms]
New \refchap{chap:paradigms}
\newcol 
\newcol \githubissue{16}, \githubissue{17}, \ref{issue:interactivity}
\newcol \Chan\ Create structure to fit a new chapter ``2. Construction paradigms'' to contain explanations 
	of different protocol paradigms for zero-knowledge proofs. 
\newcol \ref{rev:new-chapter-2-paradigms}
\rowendL
%%%%%%%%%%%%%%%%%%%%%%%%
\incItem[it:move-taxonomy-to-new-chapter][Move taxonomy to new chapter]
Old chapter 1; new \refsec{paradigms:taxonomy}
\newcol 
\newcol \githubissue{16}
\newcol \Chan\ Move the old section 1.8 (``taxonomy of constructions'') 
	to be the first section in the new paradigms chapter.
\newcol \ref{rev:move-taxonomy-to-new-chapter}
\rowendL
%%%%%%%%%%%%%%%%%%%%%%%%
\incItem[it:list-possible-paradigms][List of paradigms]
\refsec{paradigms:others}
\newcol 
\newcol \githubissue{16}, \githubissue{17}
\newcol \Chan\ List several possible ZKP protocol paradigms, each of which may later 
	become its own section with a detailed explanation of the paradigm.
\newcol \ref{rev:list-other-paradigms}
\rowendL
%%%%%%%%%%%%%%%%%%%%%%%%
\incItem[it:editorial:applications-move-notation][Types of verifiability]
Old section 3.2 (``Notation and terminology'') in chapter ``Applications''
\newcol 
\newcol \ref{it:applications:define-predicate-and-gadgets}, \ref{it:scope-use-cases-new-content}
\newcol \Note\ The section ``Notation and terminology'' was only focused on distinguishing three types of verifiability requirements.
				\Chan\ Change the section title to ``Types of verifiability'', added a header label for each enumerated type, along with minor editorial adjustments.
				Move some newly proposed definitions of gadget and predicate (see \ref{it:applications:define-predicate-and-gadgets}) to the previous introductory section (\refsec{apps:intro}).
				Edit proposed content about the scope of use-cases related to the designated verifier case (see \ref{it:scope-use-cases-new-content}).
\newcol \ref{rev:applications:define-predicate-and-gadgets}, \ref{rev:applications:move-mention-to-syntax}, \ref{rev:editorial:applications:verifiability-type}
\rowendL
%%%%%%%%%%%%%%%%%%%%%%%%
\incItem[it:editorial:tables-gadgets-portrait][Tables of gadgets]
\refsec{apps:gadgets-within-predicates}
\newcol 
\newcol \ref{it:applications:define-predicate-and-gadgets}
\newcol \Note\ Tables of individual gadgets were in pages with landscape orientation.
				\Chan\ Remove unused column ``API'' and adapt column lengths for better fit in pages with portrait orientation. A text text edits inside the cells.
\newcol \ref{rev:editorial:tables-gadgets-portrait}
\rowendL
%%%%%%%%%%%%%%%%%%%%%%%%
\incItem[it:editorial:tables-functionalities-portrait][Tables of functionality]
\refsec{sec:apps:id-framework:protocol-description}
\newcol 
\newcol 
\newcol \Note\ In the old single table of functionalities, across three landscape pages, the first column ``Functionality/problem'' spanned a large vertical space, with a short label.
				\Chan\ Converted each row defined by a ``functionality/problem'' into its own table, thus reducing the horizontal width and allowing a better fit in portrait mode.
\newcol \ref{rev:editorial:tables-functionalities-portrait}
\rowendL
%%%%%%%%%%%%%%%%%%%%%%%%
\incItem[it:editorial:remove-list-refs-in-chapters][List of used references]
End of each old chapter
\newcol Consolidate the list of used references
\newcol 
\newcol \Chan\ Remove the redundant lists of references that were remaining in the end of each chapter.
				A few of the listed references were not cited elsewhere and where thus placed were suitable.
				(The list of all references is now consolidated in a single ``\hyperref[references]{References}'' section.)
\newcol \ref{rev:editorial:remove-list-refs-in-chapter-theory}, \ref{rev:editorial:remove-list-refs-in-chapter-implementation}, \ref{rev:editorial:remove-list-refs-in-chapter-applications}
\rowendL
%%%%%%%%%%%%%%%%%%%%%%%%
\incItem[it:editorial:remove-zcon0-notes][ZCon notes (removed)]
Old chapter 4 ZCon0
\newcol 
\newcol \githubissue{16}
\newcol \Chan\ Remove the ZCon0 notes (old chapter 4). \Note\ The current editorial process separates the workshop notes from the community reference.
\newcol 
\rowendL
%%%%%%%%%%%%%%%%%%%%%%%%
\incItem[it:editorial:address-popup-annotations][PDF popup annotations]
All document
\newcol 
\newcol \githubissue{16}
\newcol \Chan\ Remove all popup pdf-annotations (done by simply clearing the definition 
				of the calling LaTeX command \textbackslash{}pdfcomment --- the comments remain 
				in the LaTeX code for future address).
\newcol 
\rowendL
%%%%%%%%%%%%%%%%%%%%%%%%
\incItem[it:editorial:add-acks][Acknowledgments]
Before the references
\newcol 
\newcol \githubissue{16}
\newcol \Chan\ Add an \hyperref[app:acknowledgments]{acknowledgments} section consistent with the contributions provided to the document.
\newcol \ref{rev:add-acks}
\rowendL
%%%%%%%%%%%%%%%%%%%%%%%%
\incItem[it:editorial:add-version-history][Version history]
Front matter
\newcol 
\newcol \githubissue{16}
\newcol \Chan\ Add a \hyperref[app:version-history]{Version history} section, with a summarized description of the sequence of main versions of the document.
\newcol \ref{rev:editorial:add-version-history}
\rowendL
%%%%%%%%%%%%%%%%%%%%%%%%
\myendIssue
