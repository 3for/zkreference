%%% This file: defines counters and commands useful for cross-referencing feedback notes and corresponding changes


\newcommand{\ifoddeven}[2]{\ifthenelse{\isodd{\value{page}}}{#1}{#2}}


\providecommand{\ifEmpty}[2]{\ifthenelse{\equal{#1}{}}{\protect #2}{}}
\providecommand{\ifEmptyElse}[3]{\ifthenelse{\equal{#1}{}}{\protect #2}{\protect #3}}
\providecommand{\ifNotEmpty}[2]{\ifthenelse{\equal{#1}{}}{}{\protect #2}}
\providecommand{\ifNotEmptyElse}[3]{\ifthenelse{\equal{#1}{}}{\protect #3}{\protect #2}}
\providecommand{\ifNotEmptyPrepend}[2]{\ifthenelse{\equal{#1}{}}{}{#2#1}}
\providecommand{\ifNotEmptyAppend}[2]{\ifthenelse{\equal{#1}{}}{}{#1#2}}



%%%%%%%%%%%% Commands
\newcommand{\tops}[2]{\texorpdfstring{#1}{#2}}
\newcommand{\nolink}[1]{\href*{#1}{#1}}

\newcommand{\subtab}[2][c]{\begin{tabular}{@{}#1}#2\end{tabular}}
\newcommand{\subtabL}[1]{\subtab[l]{#1}}

\newcommand{\edsubtab}[2][c]{\begin{edtable}{tabular}{@{}#1}#2\end{edtable}}

\newcommand{\brkttext}[1]{\ensuremath{\left\langle\right.}#1\ensuremath{\left.\right\rangle}}
\newcommand{\brktmath}[1]{\ensuremath{\left\langle\right.#1\left.\right\rangle}}

\newenvironment{bulletize}{\begin{itemize}[label={$\bullet$}]}{\end{itemize}}

\newcommand{\mypar}[1]{\paragraph{#1}\mbox{}\\\vspace{-1ex}}
\newcommand{\myurl}[1]{\href{#1}{\nolinkurl{#1}}}
%%%\newcommand{\myurl}[1]{{\hypersetup{pdfborder={0 0 0}}\url{#1}}}
\urlstyle{rm} %tm, tt, sf, ... same, ...

\newcommand{\myemail}[1]{\href{mailto:#1}{#1}}

\newcommand{\red}[1]{{\color[rgb]{1,0,0}#1}}
\newcommand{\blue}[1]{{\color[rgb]{0,0,1}#1}}


%%%%%%%%%%%% Auxi commands for longtable
\newcommand{\begmin}[2][t]{\begin{minipage}[#1]{#2\textwidth}\vspace{0pt}} %-1.5ex need to coordinate with the space in last column
\newcommand{\myendmini}{\vspace{1ex}\end{minipage}}
\newcommand{\rowend}{\end{minipage}\\}


\newcommand{\restoline}[1]{\resizebox{\linewidth}{!}{#1}}

\newcommand\cleartooddpage{\clearpage\ifodd\value{page}\else\null\clearpage\fi}

\newcommand{\shortleftarrow}{\ensuremath{\scalebox{1}{\ensuremath{\leftarrow}}}}


%LB: colors for bookmarks of table and Figures
\newcommand{\colorbkmtab}{brown}
\newcommand{\colorbkmfig}{blue}

%%%%%%%%%%%%%%%%%%%%%%%%%%%% Caption for Table, adding special pdf bookmark and allowing line number

%%% LB: \mytabcap: to be used (instead of \caption) inside environment table
  % does not require the `table' environment, but can be inside of it
  % Enables correct line numbers and it also creates a pdfbookmark
  % 1st arg is the text for the listoffigures (lot); 2nd arg is the main text of the caption
\newcommand{\mytabcap}[3][]{%
	%\addtocounter{table}{1}\hypertarget{ht:table:\thetable}{}\addtocounter{table}{-1}
	{\normalsize\centering\internallinenumbers\refstepcounter{table}\label{#1}\bfseries\hypertarget{ht:table:\thetable}{Table}~\thetable{}: %intentional space before comment %
				\addcontentsline{lot}{table}{Table~\thetable{}: #2}%  %%%\protect\ifthenelse{\equal{#1}{}}{#2}{#1}
				#3} %next there is an intentional empty line

	\nolinenumbers{\bookmarksetup{color=\colorbkmtab}%,bold=true
	\ifthenelse{\equal{\arabic{section}}{0}}{\bookmarksetup{level=1}}{\ifthenelse{\equal{\arabic{subsection}}{0}}{\bookmarksetup{level=2}}{\ifthenelse{\equal{\arabic{subsubsection}}{0}}{\bookmarksetup{level=3}}{\bookmarksetup{level=4}}}}%
	%in the line above, change the last bookmarksetup-level to 3 is wanting the pdf-bookmark to go inside the subsection
	\bookmark[dest=ht:table:\thetable]{Table~\thetable}%
	\bookmarksetup{color=black}%,bold=false
	}
}

%%%%%%%%%%%%%%%%%%%%%%%%%%%% Caption for Figure, adding special pdf bookmark

\newcommand{\myfigcap}[2]{\refstepcounter{figure}\centering{\bfseries Figure~\thefigure{}:}
														 \addcontentsline{lof}{figure}{Figure~\thefigure{}: #1} #2 \nolinenumbers}

\definecolor{colorRowHead}{rgb}{0.95,0.95,0.95}

%%%%%%%%%%%%%%%%%%%%%%%%%%%% Commands

\def\fillindesc{\red{$<$fill with description$>$}}

\newcommand{\loosen}{\looseness=-1}

\newcommand{\refchap}[1]{Chapter~\ref{#1}}
\newcommand{\refsec}[1]{Section~\ref{#1}}
\newcommand{\reftab}[1]{Table~\ref{#1}}
\newcommand{\reffig}[1]{Figure~\ref{#1}}

\newcommand{\refrem}[1]{Remark~\ref{#1}}
\newcommand{\refGad}[1]{Gadget~\ref{#1}}
\newcommand{\refgad}[1]{gadget~\ref{#1}}


\newcommand{\Comm}{\textmd{Comm}}  %%% \ensuremath{\mathcal{C}}


\newcommand{\prov}{\tops{\ensuremath{P}}{P}}
\newcommand{\veri}{\tops{\ensuremath{V}}{V}}

\newcommand{\XOR}{\texttt{XOR}}
\newcommand{\AND}{\texttt{AND}}
\newcommand{\NOT}{\texttt{NOT}}

\newcommand{\privsetP}{\tops{\ensuremath{\textmd{PrivateSetup}_{\prov}}}{PrivateSetup\_\prov}}
\newcommand{\privsetV}{\tops{\ensuremath{\textmd{PrivateSetup}_{\veri}}}{PrivateSetup\_\veri}}
\newcommand{\setP}{\tops{\ensuremath{\textmd{setup}_{\prov}}}{setup\_\prov}}
\newcommand{\setV}{\tops{\ensuremath{\textmd{setup}_{\veri}}}{setup\_\veri}}
\newcommand{\setR}{\tops{\ensuremath{\textmd{setup}_{R}}}{setup\_R}}
\newcommand{\params}{\textit{parameters}}  %LB: consider using params

\newcommand{\pslabel}{\phantomsection\label}

\newcommand{\field}{\ensuremath{\mathbb{F}}}  %needs package amssymb


\newcommand{\shall}{{\ttfamily shall}}
\newcommand{\should}{{\ttfamily should}}
\newcommand{\can}{{\ttfamily can}}
\newcommand{\neednot}{{\ttfamily need not}}


\newcommandtwoopt{\minip}[3][\textwidth][\textwidth]{\nolinenumbers\resizebox{#1}{!}{\begin{minipage}[t]{#2}\begin{internallinenumbers}#3\end{internallinenumbers}\end{minipage}}\vskip0pt\linenumbers}

\newcommand{\todo}[1]{\red{[To-do: #1]}}

\newcommand{\emphbrkt}[1]{\ensuremath{\left\langle\emph{#1}\right\rangle}}