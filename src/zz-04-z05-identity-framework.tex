%%%%%%%%%%%%%%%%%%%%%%%%%%%%%%%%%%%%%%%%%%%%%%%%%%%%%%%%%%%%
%%%%%%%%%%%%%%%%%%%%%%%%%%%%%%%%%%%%%%%%%%%%%%%%%%%%%%%%%%%%
\section{Identity framework}
\label{apps:id-framework}



%%%%%%%%%%%%%%%%%%%%%%%%%%%%%%
\subsection{Overview}
\label{sec:apps:id-framework:overview}

%\setlength{\parindent}{2em}
%\setlength{\parskip}{0pt}

In this section we describe identity management solutions using zero knowledge proofs. The idea is that some user has a set of attributes that will be attested to by an issuer or multiple issuers, such that these attestations correspond to a validation of those attributes or a subset of them. 


After attestation it is possible to use this information, hereby called a credential, to generate a claim about those attributes. Namely, consider the case where Alice wants to show that she is over 18 and lives in a country that belongs to the European Union. If two issuers were responsible for the attestation of Alice`s age and residence country, then we have that Alice could use zero knowledge proofs in order to show that she possesses those attributes, for instance she can use zero knowledge range proofs to show that her age is over 18, and zero knowledge set membership to prove that she lives in a country that belongs to the European Union. This proof can be presented to a Verifier that must validate such proof to authorize Alice to use some service. Hence there are three parties involved: (i) the credential holder; (ii) the credential issuer; (iii) and the verifier.  

  


%%%%%%%%%%%%%%%%%%%%%%%%%%%%%%
\subsection{Motivation for Identity and Zero Knowledge}
\label{sec:apps:id-framework:motivation}

Digital identity has been a problem of interest to both academics and industry practitioners since the creation of the internet. Specifically, it is the problem of allowing an individual, a company, or an asset to be identified online without having to generate a physical identification for it, such as an ID card, a signed document, a license, etc. Digitizing Identity comes with some unique risks, loss of privacy and consequent exposure to Identity theft, surveillance, social engineering and other damaging efforts. Indeed, this is something that has been solved partially, with the help of cryptographic tools to achieve moderate privacy (password encryption, public key certificates, internet protocols like TLS\luiscom{Add citation. Capitalized it.} and several others). Yet, these solutions are sometimes not enough to meet the privacy needs to the users / identities online. Cryptographic zero knowledge proofs can further enhance the ability to interact digitally and gain both privacy and the assurance of legitimacy required for the correctness of a process.
 
The following is an overview of the generalized version of the identity scheme. We define the terminology used for the data structures and the actors, elaborate on what features we include and what are the privacy assurances that we look for. 


%%%%%%%%%%%%%%%%%%%%%%%%%%%%%%
\subsection{Terminology / Definitions}
\label{sec:apps:id-framework:terminology}

In this protocol we use several different data structures to represent the information being transferred or exchanged between the parties. We have tried to generalize the definitions as much as possible, while adapting to the existing Identity standards and previous ZKP works.

\paragraph{Attribute.} The most fundamental information about a holder in the system (e.g.: age, nationality, univ. Degree, pending debt, etc.). These are the properties that are factual and from which specific authorizations can be derived.

\textbf{(Confidential and Anonymous) Credential.} The data structure that contains attribute(s) about a holder in the system (e.g.: credit card statement, marital status, age, address, etc). Since it contains private data, a credential is not shareable. 


\textbf{(Verifiable) Claim.} A zero-knowledge predicate about the attributes in a credential (or many of them). A claim must be done about an identity\luiscom{Unclear why it ``must'' be about an identity.} and should contain some form of logical statement that is included in the constraint system defined by the zk-predicate. 

\textbf{Proof of Credential.} The zero knowledge proof that is used to verify the claim attested by the credential. Given that the credential is kept confidential, the proof derived from it is presented as a way to prove the claim in question.

\vspace{1em}
\noindent The following are the different parties present in the protocol:

\textbf{Holder.} The party whose attributes will be attested to. The holder holds the credentials that contain his / her attributes and generates Zero Knowledge Proofs to prove some claim about these. We say that the holder presents a proof of credential for some claim.

\textbf{Issuer.} The party that attests attributes of holders. We say that the issuer issues a credential to the holder.

\textbf{Verifier.} The party that verifies some claim about a holder by verifying the zero knowledge proof of credential to the claim.

\vspace{1em}
\noindent\underline{Remark:} The main difference between this protocol and a non-ZK based Identity protocol is the fact that in the latter, the holder presents the credentials themselves as the proof for the claim / authorization, whereas in this protocol, the holder presents a zero knowledge proof that was computed from the credentials.


%%%%%%%%%%%%%%%%%%%%%%%%%%%%%%%%%%%%%%%%%%%%%%%%%%%%%%%%%%%%%%%%
\subsection{The Protocol Description}
\label{sec:apps:id-framework:protocol-description}


%%%%%%%%%%%%%%%%%%%%%%%%%%%%%%%%
\paragraph{Functionality.}
There are many interesting features that we considered as part of the identity protocol. 
There are four basic functionalities that we decided to include from the get go: 
\begin{itemize}
\item[(1)] third party anonymous and confidential attribute attestations through \textbf{credential issuance} by the issuer;
\item[(2)] confidentially proving claims using zero knowledge proofs through the \textbf{presentation of proof of credential} by the holder;
\item[(3)] \textbf{verification of claims} through zero knowledge proof verification by the verifier; and
\item[(4)] unlinkable \textbf{credential revocation} by the issuer. 
\end{itemize}

There are further functionalities that we find interesting and worth exploring but that we did not include in this version of the protocol. 
Some of these are credential transfer, authority delegation and trace auditability. We explain more in detail what these are and explore ways they could be instantiated. 


%%%%%%%%%%%%%%%%%%%%%%%%%%%%%%%%
\paragraph{Privacy requirements.}
One should aim for a high level of privacy for each of the actors in the system, but without compromising the correctness of the protocol. We look at anonymity properties for each of the actors, confidentiality of their interactions and data exchanges, and at the unlinkability of public data (in committed form). These usually can be instantiated as cryptographic requirements such as commitment non-malleability, indistinguishability from random data, unforgeability, accumulator soundness or as statements in zero-knowledge such as proving knowledge of preimages, proving signature verification, etc.

\begin{itemize}
		\item \underline{Holder anonymity:} the underlying physical identity\luiscom{It is unclear the actual meaning of ``physical identity''. Why does it have to be physical for the anonymity concept to apply? Consider clarifying.} of the holder must be hidden from the general public, and if needed from the issuer and verifier too. For this we use pseudo-random strings called identifiers, which are tied to a secret only known to the holder. 
		\item \underline{Issuer anonymity:} only the holder should know what issuer issued a specific credential.
		\item \underline{Anonymous credential:} when a holder presents a credential, the verifier may not know who issued the certificate. He / She may only know that the credential was issued by some approved issuer.
		\item \underline{Holder untraceability:} the holder identifiers and credentials can’t be used to track holders through time.
		\item \underline{Confidentiality:} no one but the holder and the issuer should know what the credential attributes are.
		\item \underline{Identifier linkability:} no one should be able to link two identifier unless there is a proof presented by the holder.
		\item \underline{Credential linkability:} No one should be able to link two credentials from the publicly available data. Mainly, no two issuers should be able to collude and link two credentials to one same holder by using the holder’s digital identity.
\end{itemize}



%%%%%%%%%%%%%%%%%%%%%%%%%%%%%%%%
\paragraph{In depth view.} 
For the specific instantiation of the scheme, we examine 
in \reftab{tab:apps:functionality-vs-privacy-requirements}\luised{Added reference to the table.}
the different ways that these requirements can be achieved and what are the trade-offs to be done (e.g.: using pairwise identifiers vs. one fixed public key; different revocation mechanisms; etc.) and elaborate on the privacy and efficiency properties of each. 



%%%%%%%%%%%%%%%%%%%%%%%%%%%%%%%%%%%%%%%%%%%%%%%%%%%%%%%% NEW TABLE

\newcommand{\begminB}[1]{\begmin{#1}\setlist[itemize]{leftmargin=1.5ex}\renewcommand{\labelitemi}{-}}
\newenvironment{funcprivtabular}[4]{\begin{center}\footnotesize\begin{longtable}{|>{\begminB{#1}}l<{\myendmini}|>{\begminB{#2}}l<{\myendmini}|>{\begminB{#3}}l<{\myendmini}|>{\begminB{#4}}l|}\hline }{\end{longtable}\vspace{1em}\end{center}}

\afterpage{%
    \clearpage% 
\begin{landscape} 

\vspace{.5em}
\def\tmptabtitle{Functionalities vs.\ privacy and robustness requirements}
{\mytabcap{\tmptabtitle}{\tmptabtitle\luistodo{Consider breaking this into 4 tables, each with only 3 columns}}\label{tab:apps:functionality-vs-privacy-requirements}}
\vspace{-1.5em}
\begin{funcprivtabular}{.1927}{.3327}{.3327}{.3327}
\rowcolor{colorRowHead}\textbf{Functionality / Problem}
     & \textbf{Instantiation Method}
		 & \textbf{Proof Details}
		 & \textbf{Privacy / Robustness}
		\rowend
%%%%%%%%%%%%
\hline
    \textbf{Holder identification:} how  to identify a holder of credentials
	& Single identifier in the federated realm: PRF based Public Key (idPK) derived from the physical ID\luiscom{What is the physical ID. Does it mean the person, some hardware (e.g., smartcard), ...?} of the entity and attested / onboarded by a federal authority 
	& \begin{itemize}
    \item The first credential an entity must get is the onboarding credential that attests to its identity on the system
		\item Any proof of credential generated by the holder must include a verification that the idPK was issued an onboarding credential
		\end{itemize}
	& \begin{itemize}
    \item Physical identity is hidden yet connected to the public key. 
    \item Issuers can collude to link different credentials by the same holder. 
    \item An entity can have only one identity in the system 
		\end{itemize}
	\rowend
%%%%%%%%%%%%
\cline{2-4}
	& Single identifier in the self-sovereign realm: PRF based Public Key (idPK) self derived by the entity.
	& \begin{itemize}	
    \item Any proof of credential must show the holder knows the preimage of the idPK and that the credential was issued to the idPK in question
		\end{itemize}
	& \begin{itemize}	
		\item Physical identity is hidden and does not necessarily have to be connected to the public key
		\item Issuers can collude to link different credentials by the same holder
		\item An entity can have several identities and conveniently forget any of them upon issuance of a “negative credential”
		\end{itemize}
	\rowend
%%%%%%%%%%%%
\cline{2-4}
	& Multiple identifiers: Pairwise identification through identifiers. For each new interaction the holder generates a new identifier. 
  & \begin{itemize}	
		\item Every time a holder needs to connect to a previous issuer, it must prove a connection of the new and old identifiers in ZK
    \item Any proof of credential must show the holder knows the secret of the identifier that the credential was issued to.
		\end{itemize}	 
	& \begin{itemize}	 
		\item Physical identity is hidden and does not necessarily have to be connected to the public key
    \item Issuers cannot collude to\luiscom{Editorial: issuers can always collude; the question is what capabilities they can gain therefrom. Consider: "A set of colluding issuers cannot ..."} link the credentials by the same holder
    \item An entity can have several identities and conveniently forget any of them upon issuance of a “negative credential”\luiscom{Consider clarifying the meaning of negative credential.}
		\end{itemize}
	\rowend
%%%%%%%%%%%%
\hline
	  \textbf{Issuer identification}
	& Federated permissions: there is a list of approved issuers that can be updated by either a central authority or a set of nodes
	& \begin{itemize}
    \item To accept a credential one must validate the signature against one from the list. To maintain the anonymity of the issuer, ring signatures can be used
    \item For every proof of credential, a holder must prove that the signature in its credential is of an issuer in the approved list
		\end{itemize}
	& \begin{itemize}
    \item The verifier / public would not know who the issuer of the credential is but would know it is approved.
		\end{itemize}
	\rowend
%%%%%%%%%%%%
\cline{2-4}
	& Free permissions: anyone can become an issuer, which use identifiers:
		\begin{itemize}
    \item Public identifier: type 1 is the issuer whose signature verification key is publicly available
    \item Pair-wise identifiers: type 2 is the issuer whose signature verification key can be identified only pair-wise with the holder / verifier
    \end{itemize}
	& \begin{itemize}
		\item The credentials issued by type 1 issuers can be used in proofs to unrelated parties
    \item The credentials issued by type 2 issuers can only be used in proofs to parties who know the issuer in question.
		\end{itemize}
	& \begin{itemize}
    \item If ring signatures are used, the type one issuer identifiers would not imply that the identity of the issuer can be linked to a credential, it would only mean that “Key K\_a belongs to company A”
    \item Otherwise, only the type two issuers would be anonymous and unlinkable to credentials
		\end{itemize}
	\rowend
%%%%%%%%%%%%
\hline
		\textbf{Credential Issuance}
	& Blind signatures: the issuer signs on a commitment of a self-attested credential after seeing a proof of correct attestation; a second kind of proof would be needed in the system
	& \begin{itemize}
    \item The proof of correct attestation must contain the structure, data types, ranges and credential type that the issuer allows
    \item In some cases, the proof must contain verification of the attributes themselves (e.g.: address is in Florida, but not know the city)
			\begin{itemize}
				\item The proof of credential must not be accepted if the signature of the 	credential was not verified either in zero-knowledge or as part of some public verification
			\end{itemize}
		\end{itemize}
	& \begin{itemize}
    \item Issuer’s signatures on credentials add limited legitimacy: a holder could add specific values / attributes that are not real and the issuer would not know
    \item An Issuer can collude with a holder to produce blind signatures without the issuer being blamed
		\end{itemize}
	\rowend
%%%%%%%%%%%%
\cline{2-4}
	& In the clear signatures: the issuer generates the attestation, signing the commitment and sending the credential in the clear to the holder 
	& \begin{itemize}
		\item The proof of credential must not be accepted if the signature of the credential was
	not verified either in zero-knowledge or as part of some public verification
		\end{itemize}
	& \begin{itemize}
    \item Issuer must be trusted, since she can see the Holder’s data and could share it with others
    \item The signature of the issuer can be trusted and blame could be allocated to the issuer
		\end{itemize}
	\rowend
%%%%%%%%%%%%
\hline
		\textbf{Credential Revocation}
	& Positive accumulator revocation: the issuer revokes the credential by removing an element from an accumulator
	& \begin{itemize}
		\item The holder must prove set membership of a credential to prove it was issued and was not revoked at the same time
		\item The issuer can revoke a credential by removing the element that represents it from the accumulator
		\end{itemize}
	& \begin{itemize}
		\item If the accumulator is maintained by a central authority, then only the authority can link the revocation to the original issuance, avoiding timing attacks by general parties (join-revoke linkability)
		\item If the accumulator is maintained through a public state, then there can be linkability of revocation with issuance since one can track the added values and test its membership
		\end{itemize}
	\rowend
%%%%%%%%%%%%
\cline{2-4}
	& Negative accumulator revocation: the issuer revokes by adding an element to an accumulator
	& \begin{itemize}
		\item The holder must prove set membership of a credential to prove it was issued
		\item The issuer can revoke a credential by adding to the negative accumulator the revocation secret related to the credential to be revoked
		\item The holder must prove set non-membership of a revocation secret 	
		associated to the credential in question
		\item The verifier must use the most recent version of the accumulator to validate the claim
		\end{itemize}
	& \begin{itemize}
		\item Even when the accumulator is maintained through a public state, the revocation cannot be linked to the issuance since the two events are independent of each other
		\end{itemize}
	\rowend
%%%%%%%%%%%%
\hline
\end{funcprivtabular}

\end{landscape}
} %end of \afterpage

\paragraph{Gadgets.}

Each of the methods for instantiating the different functionalities use some of the following gadgets that have been described in the Gadgets section. There are three main parts to the predicate of any proof.
\begin{enumerate}
\item The first is proving the veracity of the identity, in this case the holder, for which the following gadgets can / should be used:
		\begin{itemize}
		\item \textbf{Commitment} for checking that the identity has been attested 			to correctly.
		\item \textbf{PRF} for proving the preimage of the identifier is known by the holder
		\item \textbf{Equality of strings} to prove that the new identifier has a connection to the previous identifier used or to an approved identifier.\
		\end{itemize}

\item Then there is the part of the constraint system that deals with the legitimacy of the credentials, the fact that it was correctly issued and was not revoked. 	
		\begin{itemize}
		\item \textbf{Commitment} for checking that the credential was correctly committed to.	
		\item \textbf{PRF} for proving that the holder knows the credential information, which is the preimage of the commitment .
		\item \textbf{Equality of strings} to prove that the credential was issued to an identifier connected to the current identifier.
		\item \textbf{Accumulators (Set membership / non-membership)} to prove that the commitment to the credential exists in some set (usually an accumulator), implying that it was issued correctly and that it was not revoked.
		\end{itemize}

\item Finally there is the logic needed to verify the rules / constraints imposed on the attributes themselves. This part can be seen as a general gadget called “credentials”, which allows to verify the specific attributes embedded in a credential. Depending on the 	credential type, it uses the following low level gadgets:	
		\begin{itemize}
		\item \textbf{Data Type} used to check that the data in the credential is of the correct type 	
		\item \textbf{Range Proofs} used to check that the data in the credential is within some range
		\item \textbf{Arithmetic Operations (field arithmetic, large integers, etc.)} used for verifying arithmetic operations were done correctly in the computation of the instance.
		\item \textbf{Logical Operators (bigger than, equality, etc.)} used for comparing some value in the instance to the data in the credentials or some computation derived from it.
		\end{itemize}
\end{enumerate}



%%%%%%%%%%%%%%%%%%%%%%%%%%%%%%%%
\paragraph{Security caveats}

\begin{enumerate}
	\item If the Issuer colludes with the Verifier, they could use the revocation mechanism to reveal information about the Holder if there is real-time sharing of revocation information.
	\item Furthermore, if the commitments to credentials and the revocation 	information can be tracked publicly and the events are dependent of each other (e.g.: revocation by removing a commitment), then there can be linkability between issuance and revocation.	
	\item In the case of self-attestation or collusion between the issuer and the holder, there is a much lower assurance of data integrity. The inputs to the ZKP could be spoofed and then the proof would not be sound.
	\item The use of Blockchains create a reliance on a trusted oracle for external state. On the other hand, the privacy guaranteed at blockchain-content level is orthogonal to network-level traffic analysis.	
\end{enumerate}



%%%%%%%%%%%%%%%%%%%%%%%%%%%%%%%%%%%%%%%%%%%%%%%%%%%%%%%5
\subsection{A use-case example of credential aggregation}
\label{sec:apps:id-framework:use-case-credential-aggregation}


\revblock[rev:apps:identity-framework:move-paragraph-focus-accredited-investors]{it:apps:identity-framework:move-paragraph-focus-accredited-investors}
We are going to focus our description on a specific use case: accredited investors. 
In this scenario the credential holder will be able to show that she is accredited without revealing more information than necessary to prove such a claim. 


\paragraph{Use-case description.}
As a way to illustrate the above protocol, we present a specific use-case and explicitly write the predicate of the proof. Mainly, there is an identity, Alice, who wants to prove to some company, Bob Inc. that she is an accredited investor, under the SEC rules, in order to acquire some company shares. Alice is the prover; the IRS, the AML entity and The Bank are all issuers; and Bob Inc. is the verifier.

The different processes in the adaptation of the use-case are the following:

\begin{enumerate}
\item Three confidential credentials are issued to Alice which represent the rules that we apply on an entity to be an accredited investor\footnote{We assume that the SEC generates the constraint system for the accreditation rules as the circuit used to generate the proving and verification keys. In the real scenario, here are the \href{https://www.ecfr.gov/cgi-bin/retrieveECFR?gp=&SID=8edfd12967d69c024485029d968ee737&r=SECTION&n=17y3.0.1.1.12.0.46.176}{Federal Rules for accreditation}.}:
		\begin{enumerate}
        \item The IRS issues a tax credential, $C_0$, that testifies to the claim “from 1/1/2017 until 1/1/2018, Alice, with identifier $X_0$, ows 0\$ to the IRS, with identifier $Y$” and holds two attributes: the net income of Alice, \$income, and a bit $b$ such that $b=1$ if Alice has paid her taxes.
        \item The AML entity issues a KYC credential, $C_1$, that testifies to claim $T_1$:= “Alice, with identifier $X_1$, has NO relation to a (set of) blacklisted organization(s)”
        \item The Bank issues a net-worth credential, $C_2$, that testifies to claim $T_2$:= “Alice has a net worth of $V\textsubscript{Alice}$”
		\end{enumerate}
\item Alice then proves to Bob Inc. that:
    \begin{enumerate}
				\item “Alice’s identifier, $X\textsubscript{Bob}$, is related to the identifiers, {$X_i$} for $i = 0, 1, 2$ that are connected to the confidential credentials {$C_i$}”
        \item “I know the credentials, which are the preimage of some commitment, {$C_i$}, were issued by the legitimate issuers”
        \item “The credentials, which are the preimage of some commitment, {$C_i$}, that exist in an accumulator, $U$, satisfy the three statements {$T_i$}”
		\end{enumerate}
\end{enumerate}



\paragraph{Instantiation details.}
Based on the different options laid out in the table above, the following have been used:

\begin{itemize}
    \item \underline{Holder identification:} we instantiate the identifiers as a unique anonymous identifier, publicKey
    \item \underline{Issuance identification:} the identity of the issuers is known to all the participants, who can publicly verify the signature on the credentials they issue\footnote{With public signature verification keys that are hard coded into the circuit}.
    \item \underline{Credential issuance:} credentials are issued by publishing a signed commitment to a positive accumulator and sharing the credential in the clear to Alice.
    \item \underline{Credential revocation:} is done by removing the commitment of credential from a dynamic and positive accumulator. 
		Alice must prove membership of commitment to show her credential was not revoked.
    \item \underline{Credential verification:} Bob Inc. then verifies the cryptographic proof with the instance.
\end{itemize}


\mbox{}\\
Note that the transfer of company shares as well as the issuance of company shares is outside of the scope of this use-case, but one could use the “Asset Transfer” section of this document to provide that functionality. 

On another note, the fact that the proving and verification keys were validated by the SEC is an assurance to Bob Inc. that proof verification implies Alice is an accredited investor.


\paragraph{The Predicate}
\begin{itemize}
\item \blue{Blue = publicly visible in protocol / statement}
\item \red{Red = secret witness, potentially shared between parties when proving}
\end{itemize}



%%%%%%%%%%%%%%%%%%%%%%
\textbf{Definitions / Notation:}


Public state: \blue{Accumulator}, for issuance and revocation, which includes all the commitments to the credentials.


\vspace{.5em}
\red{ConfCred} = Commitment to Cred = \{ \blue{Revoke}, \red{certificateType, publicKey, Attribute(s)}  \}

Where, again, the IRS, AML and Bank are authorities with well-known public keys. Alice’s \red{publicKey} is her long term public key and one cannot create a new credential unless her long term ID has been endorsed. The goal of the scheme is for the holder to create a fresh \textbf{proof of confidential aggregated credentials to the claim of accredited investor}.

\def\subIRS{\textsubscript{IRS}}
\def\subAML{\textsubscript{AML}}
\def\subrhIRS{\textsubscript{rhIRS}}
\def\subrhAML{\textsubscript{rhAML}}


IRS issues a \red{ConfCred\subIRS} = Commitment( openIRS, revokeIRS, “IRS”, myID, \$Income, b ), sigIRS
AML issues \red{ConfCred\subAML}= Commitment( openAML, revokeAML, “AML”, myID, “OK”), sigAML

Holder generates a fresh public key \blue{freshCred} to serve as an ephemeral blinded aggregate credential, and a ZKP of the following:

\textbf{ZkPoK}\{  (witness: \red{myID, ConfCred\subIRS, ConfCred\subAML, sigIRS, sigAML, \$Income, , mySig, open\subIRS, open\subAML} statement: \blue{freshCred, minIncomeAccredited} ) :  
Predicate: 
	\begin{itemize}[label={- }]
		\item \red{ConfCred\subIRS} is a commitment to the IRS credential  ( \red{open\subIRS}, \blue{“IRS”}, \red{myID, \$Income} )
		\item \red{ConfCred\subAML} is the AML crdential to ( \red{open\subAML}, \blue{“AML”}, \red{myID}, \blue{“OK”} )
		\item \red{\$Income} >= \blue{minIncomeAccredited}
		\item \red{b} = 1 = “myID paid full taxes”
		\item \red{mySig} is a signature on \blue{freshCred} for \red{myID}
		\item \textbf{ProveNonRevoke(  )}
	\end{itemize}

\hphantom{\textbf{ZkPoK}}\}


Present the credential to relying party:\luiscom{The term ``relying party'' requires prior definition.}
\blue{freshCred} and \textbf{zkp}.



\vspace{.5em}\textbf{ProveNonRevoke}( rhIRS, w\_hrIRS, rhAML, w\_hrAML, a\_IRS
\begin{itemize}
    \item \red{revokeIRS}: revocation handler from IRS. Can be embedded as an attribute in \red{ConfCredt\subIRS} and is used to handle revocations.
    \item \red{wit\subrhIRS}: accumulator witness of \red{revokeIRS}.
    \item \red{revokeAML}: revocation handler from AML. Can be embedded as an attribute in \red{ConfCredt\subAML} and is used to handle revocations.
    \item \red{wit\subrhAML}: accumulator witness of \red{revokeAML}.
    \item \blue{acc\subIRS}: accumulator for IRS.
    \item \blue{CommRevoke\subIRS}: commitment to \red{revokeIRS}. The holder generates a new commitment for each revocation to avoid linkability of proofs. 
    \item \blue{acc\subAML}: accumulator for AML.
    \item \blue{CommRevoke\subAML}: commitment to \red{revokeAML}. The holder generates a new commitment for each revocation to avoid linkability of proofs. 
\end{itemize}


\vspace{.5em}
\textbf{ZkPoK}\{ (witness: \red{rhIRS, open\subrhIRS, w\subrhIRS, rhAML, open\subrhAML, w\subrhAML}|| statements: \blue{C\subIRS, a\subIRS , C\subAML , a\subAML} ):
Predicate:
	\begin{itemize}[label={- }]
	\item \blue{C\subIRS} is valid commitment to ( \red{open\subrhIRS, rhIRS} )
	\item \red{rhIRS} is part of accumulator \red{a\subIRS}, under witness \red{w\subrhIRS}
	\item \red{rhIRS} is an attribute in \red{Cert\subIRS}
	\item \blue{C\subAML} is valid commitment to ( \red{open\subrhAML, rhAML} )
	\item \red{rhAML} is part of accumulator \blue{a\subAML}, under witness \red{w\subrhAML}
	\item \red{rhAML} is an attribute in \red{Cert\subAML}
	\end{itemize}
\hphantom{\textbf{ZkPoK}}\}

	\begin{itemize}
		\item[- ] myCred is unassociated with myID, with sigIRS, sigAML etc.
    \item[- ] Withstands partial compromise: even if IRS leaks myID and sigIRS, it cannot be used to reveal the sigAML or associated myID with myCred
	\end{itemize}
