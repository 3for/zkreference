%%% DO NOT LEAVE ANY EMPTY SPACES BETWEEN \newcol COMMANDS of each \newIssue, because this is parsed inside tha tabular environment
%%% Command \newIssue{label}{issue title}     % to start a new section of comments
%%% Command \incItem[optional-label]          % start a new row/comment item related to an issue
%%% Command \newcol for new columns
%%% Command \rowend for end of row

%%%For each \incItem[label], 
	%( the following initial columns are automatic: \#, ref)
	% old location % before the first \newcol
	% Proposed contribution (propoContrib) and Proposed contributors (\contributors)
	% Related
	% Context (\ccontext) and Change (\Chan)
	% Rev (Revision references)



%%%%%%%%%%%%%%%%%%%%%%%%%%%%%%%%%%%%%%%%%%%%%%%%%%%%%%%%%%%%%%%%%%%%%%%%%%%%%%%%%%%%%%%%%%%%%%%%
%%%%%%%%%%%%%%%%%%%%%%%%%%%%%%%%%%%%%%%%%%%%%%%%%%%%%%%%%%%%%%%%%%%%%%%%%%%%%%%%%%%%%%%%%%%%%%%%
\section*{Structural changes}
\addcontentsline{toc}{section}{Structural changes}

%%% Inherently related to the editorial development of the reference document

%%%%%%%%%%%%%%%%%%%%%%%%%%%%%%%%%%%%%%%%%%%%%%%%%%%%%%%%%%%%%%%%%%%%%%%%%%%%%%%%%%%%%%%%%%%%%%%%
\newIssue{issue:editorial-structural}{Implement editorial structural changes} %C16
%%%%%%%%%%%%%%%%%%%%%%%%
\incItem[it:editorial-structural]
All document
\newcol \propContrib\ Implement editorial structural changes to the document. 
				\contributors\ The editors (Daniel Benarroch, Luís Brandão, Eran Tromer) will consider the necessary structural changes (new chapters, sections, subsections, etc.) based on the overall set of contributions.
\newcol \githubissue{16}
\newcol \ccontext\ Inherently related to the editorial development of the reference document.  %%% Inherently related to the editorial development of the reference document
				\Chan\ ...
\newcol %\ref{...}
\rowendL
%%%%%%%%%%%%%%%%%%%%%%%%
\incItem[it:new-chapter-2]
New chapter 2
\newcol \propContrib\ 
		Create new chapter ``2. Construction paradigms'' to contain explanations 
	of different protocol paradigms for zero-knowledge proofs. 
	%%%\contributors\ Luís Brandão
\newcol \githubissue{16}
\newcol \ccontext\ editorial  
				\Chan\ Done as suggested
\newcol \ref{rev:new-chapter-2-paradigms}
\rowendL
%%%%%%%%%%%%%%%%%%%%%%%%
\incItem[it:move-taxonomy-to-new-chapter]
All document
\newcol \propContrib\ Move the old section 1.8 (``taxonomy of constructions'') 
	to be the first section in the new paradigms chapter.
	%%%\contributors\ Luís Brandão
\newcol \githubissue{16}, \ref{issue:interactivity}
\newcol \ccontext\ editorial  
				\Chan\ Done as suggested
\newcol \ref{rev:move-taxonomy-to-new-chapter}
\rowendL
%%%%%%%%%%%%%%%%%%%%%%%%
\incItem[it:list-possible-paradigms]
All document
\newcol \propContrib\ 
	List several possible ZKP protocol paradigms, each of which may later 
	become its own section with a detailed explanation of the paradigm.
\newcol \githubissue{16}, \githubissue{17}
\newcol \ccontext\ editorial  
				\Chan\ Done as suggested
\newcol \ref{rev:list-other-paradigms}
\rowendL
%%%%%%%%%%%%%%%%%%%%%%%%
\incItem[it:editorial:add-abstract]
Front matter, after the cover
\newcol \propContrib\ Add an abstract
				%%%\contributors\ Editors
\newcol \githubissue{16}
\newcol \ccontext\ editorial  %%% Inherently related to the editorial development of the reference document
				\Chan\ ...
\newcol \ref{rev:add-abstract}
\rowendL
%%%%%%%%%%%%%%%%%%%%%%%%
\incItem[it:editorial:add-editors-note]
Front matter
\newcol \propContrib\ Add an editorial note explaining the versioning of the document
				%%%\contributors\ Editors
\newcol \githubissue{16}
\newcol \ccontext\ editorial  %%% Inherently related to the editorial development of the reference document
				\Chan\ ...
\newcol \ref{rev:add-editorial-note}
\rowendL
%%%%%%%%%%%%%%%%%%%%%%%%
\incItem[it:editorial:add-acks]
Front matter
\newcol \propContrib\ Add acknowledgments consistent with all the contributions
				%%%\contributors\ Editors
\newcol \githubissue{16}
\newcol \ccontext\ editorial  %%% Inherently related to the editorial development of the reference document
				\Chan\ ...
\newcol \ref{rev:add-acks}
\rowendL
%%%%%%%%%%%%%%%%%%%%%%%%
\incItem[it:editorial:remove-zcon0-notes]
Old chapter 4 ZCon0
\newcol \propContrib\ Remove ZCon0 notes. Based on the editorial process, 
				the workshop notes are separated from the community reference.
				%%%\contributors\ Editors
\newcol \githubissue{16}
\newcol \ccontext\ editorial  %%% Inherently related to the editorial development of the reference document
				\Chan\ Removed the old chapter 4, which contained informal notes from the ZCon0 workshop.
\newcol 
\rowendL
%%%%%%%%%%%%%%%%%%%%%%%%
\incItem[it:editorial:address-popup-annotations]
All document
\newcol \propContrib\ Address and remove all popup pdf-annotations present in draft version 0.1.
				%%%\contributors\ Luís Brandão
\newcol \githubissue{16}
\newcol \ccontext\ editorial  %%% Inherently related to the editorial development of the reference document
				\Chan\ 
\newcol 
\rowendL
\end{longtable} %the beginning of longtable is inserted by each new \newIssue





%%%%%%%%%%%%%%%%%%%%%%%%%%%%%%%%%%%%%%%%%%%%%%%%%%%%%%%%%%%%%%%%%%%%%%%%%%%%%%%%%%%%%%%%%%%%%%%%
%%%%%%%%%%%%%%%%%%%%%%%%%%%%%%%%%%%%%%%%%%%%%%%%%%%%%%%%%%%%%%%%%%%%%%%%%%%%%%%%%%%%%%%%%%%%%%%%
\section*{New or adapted content}
\addcontentsline{toc}{section}{New or adapted content}


\begin{longtable}{l}
%%%%%%%%%%%%%%%%%%%%%%%%%%%%%%%%%%%%%%%%%%%%%%%%%%%%%%%%%%%%%%%%%%%%%%%%%%%%%%%%%%%%%%%%%%%%%%%%
\newIssue{issue:exec-summ}{Add an executive summary} %C1
%%%%%%%%%%%%%%%%%%%%%%%%
\incItem[it:exec-summ:add]
Preamble of the document, before the table of contents
\newcol \propContrib\ Include an "executive summary" describing at a high level the structure and content of the overall "ZKProof community reference" document; the new text may also allude to the purpose, aim, scope and format of the document.
				\contributors\ NIST-PEC team (Luís Brandão, René Peralta, Angela Robinson)
\newcol \githubissue{1}
\newcol \ccontext NIST comments C5, D1-D5 %%% RELATED
				\Chan\ Adding a new executive summary
\newcol \ref{rev:new-exec-summ}
\rowendL




%%%%%%%%%%%%%%%%%%%%%%%%%%%%%%%%%%%%%%%%%%%%%%%%%%%%%%%%%%%%%%%%%%%%%%%%%%%%%%%%%%%%%%%%%%%%%%%%
\newIssue{issue:clarify-pok}{Clarify proofs of knowledge} %C2
%%%%%%%%%%%%%%%%%%%%%%%%
\incItem[it:pok:describe] 
Sections 1.1 and 1.5.3
\newcol \propContrib\ Make a clearer distinction of ZK proofs of membership vs. ZK proofs of knowledge, including by means of examples and definitions; clarify how the formalism can adequately model proofs of knowledge; may also include a definition of ``extractability'' property/game.
				\contributors\ NIST-PEC team (Luís Brandão, René Peralta, Angela Robinson)
\newcol \githubissue{2}
\newcol \ccontext NIST comments C7
				\Chan\ Adding paragraph distinguishing Proofs of membership vs. proofs of knowledge; add definition game for extractability.
\newcol %\ref{rev:acronym-ZKP}, \ref{rev:ZKP:clarify-secret-info}, \ref{rev:pok:describe}
\rowendL
%%%%%%%%%%%%%%%%%%%%%%%%
\incItem[it:pok:zkp-acronym] 
Sections 1.1
\newcol Introduce acronym ZKP
\newcol \newcol 
\newcol \ref{rev:ZKP:clarify-secret-info}
\rowendL
%%%%%%%%%%%%%%%%%%%%%%%%
\incItem[it:pok:meaning-secrecy-for-prover] 
Sections 1.1
\newcol Clarify the meaning of ``secrecy'' of the ``information'' held by the prover.
\newcol \newcol 
\newcol \ref{rev:ZKP:clarify-secret-info}
\rowendL
%%%%%%%%%%%%%%%%%%%%%%%%
\incItem[it:pok:list-basic-examples] 
Sections 1.1
\newcol Enumerate the basic examples, including two new ones (chess and theorem)
\newcol \newcol 
\newcol \ref{rev:ZKP:basic-examples}
\rowendL
%%%%%%%%%%%%%%%%%%%%%%%%
\incItem[it:pok:need-instance] 
Sections 1.1
\newcol Allude to the need of an instance
\newcol \newcol 
\newcol \ref{rev:ZKP:need-instance}
\rowendL
%%%%%%%%%%%%%%%%%%%%%%%%
\incItem[it:ZKP:proof-vs-argument] 
Sections 1.1
\newcol Mention proof vs. argument
\newcol \newcol 
\newcol \ref{rev:ZKP:proof-vs-argument}
\rowendL
%%%%%%%%%%%%%%%%%%%%%%%%
\incItem[it:ZKP:enhance-table-of-examples] 
Sections 1.2
\newcol Enhance table of examples
\newcol \newcol 
\newcol \ref{rev:ZKP:enhance-table-of-examples}
\rowendL
%%%%%%%%%%%%%%%%%%%%%%%%
\incItem[it:pok:types-of-statement] 
Sections 1.3
\newcol Distinguish types of statement: of knowledge vs.\ of membership
\newcol \newcol 
\newcol \ref{rev:pok:types-of-statement}
\rowendL
%%%%%%%%%%%%%%%%%%%%%%%%
\incItem[it:pok:ZKPoK-vs-ZKPoM] 
(New) Sections 1.4
\newcol Dsitinguish types of proof: of knowledge vs.\ of membership
\newcol \newcol 
\newcol \ref{rev:pok:ZKPoK-vs-ZKPoM}
\rowendL
%%%%%%%%%%%%%%%%%%%%%%%%
\incItem[it:pok:example-ZKPoK-DL] 
(New) Sections 1.4.1
\newcol Add example of ZKPoK of DL
\newcol \newcol 
\newcol \ref{rev:pok:example-ZKPoK-DL}
\rowendL
%%%%%%%%%%%%%%%%%%%%%%%%
\incItem[it:pok:example-ZKPoK-pre-image] 
(New) Sections 1.4.2
\newcol Add example of ZKPoK of hash pre-image
\newcol \newcol 
\newcol \ref{rev:pok:example-ZKPoK-pre-image}
\rowendL
%%%%%%%%%%%%%%%%%%%%%%%%
\incItem[it:pok:example-ZKPoM-GNI] 
(New) Sections 1.4.3
\newcol Add example of ZKP of graph non-isomorphism
\newcol \newcol 
\newcol \ref{rev:pok:example-ZKPoM-GNI}
\rowendL
%%%%%%%%%%%%%%%%%%%%%%%%
\incItem[it:pok:suggest-ZKPoK-game] 
Section 1.5
\newcol Add suggestion to define ZKPoK game
\newcol \newcol 
\newcol \ref{rev:pok:suggest-ZKPoK-game}
\rowendL



%%%%%%%%%%%%%%%%%%%%%%%%%%%%%%%%%%%%%%%%%%%%%%%%%%%%%%%%%%%%%%%%%%%%%%%%%%%%%%%%%%%%%%%%%%%%%%%%
\newIssue{issue:explain-parameter-kappa}{Explain the computational security parameter} %C3
%%%%%%%%%%%%%%%%%%%%%%%%
\incItem[it:explain-parameter-kappa]
Chapter 2 ("Implementation"), mostly in Section 2.5.
\newcol \propContrib\ Add text about possible computational security parameters, and the different security properties they may apply to (e.g., soundness, ZK, short-term vs. long-term, ...). In section 2.5, replace occurrences of "120" by "128".
				\contributors\ The NIST-PEC team (Luís Brandão, René Peralta, Angela Robinson).
\newcol \githubissue{3}
\newcol \ccontext\ Proposed in the "NIST comments on the initial ZKProof documentation" (April 06, 2019) --- item C18.
				\Chan\ ...
\newcol \ref{rev:sec-par-bench-refer-to-section}
\rowendL
%%%%%%%%%%%%%%%%%%%%%%%%
\incItem[it:comp-sec-par:120-to-128] 
Section 1.5
\newcol Wrt to required (approximate) level of security, change 120 to 128
\newcol \newcol 
\newcol \ref{rev:comp-sec-par:120-to-128-a}, \ref{rev:comp-sec-par:120-to-128-b}
\rowendL
%%%%%%%%%%%%%%%%%%%%%%%%
\incItem[it:comp-sec-par:bench:characterize-properties] 
Section 1.7.1
\newcol In benchmarks, characterize different security properties
\newcol \newcol 
\newcol \ref{rev:comp-sec-par:bench:characterize-properties}
\rowendL
%%%%%%%%%%%%%%%%%%%%%%%%
\incItem[it:comp-sec-par:bench:security-levels] 
Section 1.7.2
\newcol Computational security levels for benchmarks
\newcol \newcol 
\newcol \ref{rev:comp-sec-par:bench:security-levels}, \ref{rev:comp-sec-par:exception-lower-levels}
\rowendL




%%%%%%%%%%%%%%%%%%%%%%%%%%%%%%%%%%%%%%%%%%%%%%%%%%%%%%%%%%%%%%%%%%%%%%%%%%%%%%%%%%%%%%%%%%%%%%%%
\newIssue{issue:clarify-C-in-CRS}{Clarify the public vs. non-public aspect of ``common'' in CRS enhancement} %C4
%%%%%%%%%%%%%%%%%%%%%%%%
\incItem[it:clarify-C-in-CRS] 
Mostly in Chapter 1, starting in section 1.2; will also check for other applicable cases across the document.
\newcol \propContrib\ Clarify the distinction between common (as in shared between prover and verifier) and public knowledge (as in known externally). The lack of distinction was noticed in several parts of the document, when thinking of a comparison between transferable vs. non-transferable ZK proofs. CRS is sometimes being defined as public, although in practice it could be obtained as common to the intervening parties, yet private to a particular interaction. For example, line 177 says ``common public input'' when first talking of a "common reference string", but the ``public'' aspect is arguable – being public vs. common-but-not-public may make the difference between transferability vs. non-transferability.
				\contributors\ NIST-PEC team (Luís Brandão, René Peralta, Angela Robinson).
\newcol \githubissue{4}
\newcol \ccontext\ proposed in the "NIST comments on the initial ZKProof documentation" (April 06, 2019) --- item C11.
				\Chan\ ...
\newcol %\ref{...}
\rowendL
%%%%%%%%%%%%%%%%%%%%%%%%
\incItem[it:syntax-setup] 
Section 1.2
\newcol Syntax of setup --- common and private components
\newcol \newcol 
\newcol \ref{rev:syntax-setup}
\rowendL



%%%%%%%%%%%%%%%%%%%%%%%%%%%%%%%%%%%%%%%%%%%%%%%%%%%%%%%%%%%%%%%%%%%%%%%%%%%%%%%%%%%%%%%%%%%%%%%%
\newIssue{issue:intellectual-property}{Mention intellectual property} %C5
%%%%%%%%%%%%%%%%%%%%%%%%
\incItem[it:intellectual-property]
Preamble
\newcol \propContrib\ Present (in one or two paragraphs), in a non-legalese way, several remarks about intellectual property (IP). A main goal is to raise awareness about the role that IP may take or might not take in the adoption of recommendations and requirements in the community reference document. We are aware this is a delicate topic, so a goal of the contribution is to also motivate future constructive discussion/consideration by the ZKProof community, e.g., about open-source, IP rights, reasonable and non-discriminatory IP terms, etc.
				\contributors\ NIST-PEC team (Luís Brandão, René Peralta, Angela Robinson).
\newcol \githubissue{5}
\newcol \ccontext\ Proposed in the ``NIST comments on the initial ZKProof documentation'' (April 06, 2019) --- item C22.
				\Chan\ Added a new section entitled ``Expectations on disclosure and licensing of intellectual property''
\newcol \ref{rev:intellectual-property}
\rowendL
%%%%%%%%%%%%%%%%%%%%%%%%


%%%%%%%%%%%%%%%%%%%%%%%%%%%%%%%%%%%%%%%%%%%%%%%%%%%%%%%%%%%%%%%%%%%%%%%%%%%%%%%%%%%%%%%%%%%%%%%%
\newIssue{issue:transferability}{Discuss transferability and deniability} %C6
%%%%%%%%%%%%%%%%%%%%%%%%
\incItem[it:transferability-vs-interactivity-elaborate]
Chapter 1 ("Security/Theory")
\newcol \propContrib\ Elaborate more on the concept of transferability. For example, in an interactive protocol over the Internet, how do regular authenticated channels vs. ``ideally'' authenticated channels affect transferability? Would a non-transferable protocol become transferable when the prover signs all sent messages and the verifier uses the output of a cryptographic hash function to select random challenges? 
				\contributors\ Luís Brandão
\newcol \githubissue{6}, \ref{it:deniability}
\newcol \ccontext\ Proposed in the "NIST comments on the initial ZKProof documentation" (April 06, 2019) --- item C9.
				\Chan\ ...
\newcol \ref{rev:introduce-deniability-transferability}, \ref{rev:deniability-transferability-nuances}
\rowendL
%%%%%%%%%%%%%%%%%%%%%%%%
\incItem[it:transferability-vs-interactivity-incorrect]
Old Section 3.2.
\newcol \propContrib\ In Section 3.2, revise the incorrect assertion in item 1: ``Only non-interactive ZK (NIZK) can actually hold this property'' [being publicly verifiable / transferable?]. For example, if transferability is a design goal then there are settings where it is possible to design interactive protocols for which the view (transcript) of the original verifier (interacting with the original prover) can later serve as a transferable proof for other verifiers.
				\contributors\ Luís Brandão, 
\newcol \githubissue{6}
\newcol \ccontext\ Proposed in the "NIST comments on the initial ZKProof documentation" (April 06, 2019) --- item C14.
				\Chan\ Removed sentence
\newcol \ref{rev:remove-incorrection-on-transferability}
\rowendL
%%%%%%%%%%%%%%%%%%%%%%%%
\incItem[it:deniability]
...
\newcol \propContrib\ Elaborate more on the concept of deniability.
				\contributors\ Ivan Visconti
\newcol \githubissue{6}, \ref{it:transferability-vs-interactivity-elaborate}
\newcol \ccontext\ The ``deniability'' item was identified in the breakout session on ``Interactive Zero Knowledge'' in the 2nd ZKProof workshop.
				\Chan\ ...
\newcol \ref{rev:deniability}
\rowendL



%%%%%%%%%%%%%%%%%%%%%%%%%%%%%%%%%%%%%%%%%%%%%%%%%%%%%%%%%%%%%%%%%%%%%%%%%%%%%%%%%%%%%%%%%%%%%%%%
\newIssue{issue:enhance-the-glossary}{Enhance the glossary} %C7
%%%%%%%%%%%%%%%%%%%%%%%%
\incItem[it:glossary-enhancement]
Section "A.2 Glossary" in the Appendix.
\newcol \propContrib\ Add to the glossary the suitable technical terms that are used across the document; include hyperlinks to at least its first use in the document.
				\contributors\ The editors (Daniel Benarroch, Luís Brandão, Eran Tromer).
\newcol \githubissue{7}
\newcol \ccontext\ proposed in the "NIST comments on the initial ZKProof documentation" (April 06, 2019) --- item C4.
				\Chan\ ...
\newcol %\ref{...}
\rowendL
%%%%%%%%%%%%%%%%%%%%%%%%


%%%%%%%%%%%%%%%%%%%%%%%%%%%%%%%%%%%%%%%%%%%%%%%%%%%%%%%%%%%%%%%%%%%%%%%%%%%%%%%%%%%%%%%%%%%%%%%%
\newIssue{issue:index-examples}{Index and highlight the running examples} %C8
%%%%%%%%%%%%%%%%%%%%%%%%
\incItem[it:index-examples]
Edits across the document; then include "list of recommendations" and "list of requirements" after the toc, lof and lot.
\newcol \propContrib\ Identify in the document which examples can be placed within a boxed environment, with a caption, explanation (possibly an illustration) and a footnote identifying the included concepts (e.g., ``setup, trapdoor, CRS, prover and verifier''). Some of these items may be placed as placeholders, as a way to request further contribution by the community.
				\contributors\ The editors (Daniel Benarroch, Luís Brandão, Eran Tromer) will deal with uniformizing the Latex environment for boxed examples. 
				%Anyone else please contribute in reply to this post with a description of useful examples to include.
\newcol \githubissue{8}
\newcol \ccontext\ proposed in the "NIST comments on the initial ZKProof documentation" (April 06, 2019) --- item C6.
				\Chan\ ...
\newcol %\ref{...}
\rowendL
%%%%%%%%%%%%%%%%%%%%%%%%


%%%%%%%%%%%%%%%%%%%%%%%%%%%%%%%%%%%%%%%%%%%%%%%%%%%%%%%%%%%%%%%%%%%%%%%%%%%%%%%%%%%%%%%%%%%%%%%%
\newIssue{issue:highlight-rec-and-req}{Highlight the recommendations and requirements} %C9
%%%%%%%%%%%%%%%%%%%%%%%%
\incItem[it:highlight-rec-and-req]
Edit across the document; then include "list of recommendations" and "list of requirements" after the toc, lof and lot.
\newcol \propContrib\ Review the document to identify which statements correspond to requirements and/or recommendations. Create a proper indexed latex environment to highlight these requirements and/or recommendations. It is expected that some of these items will be edited in a tentative manner, as they may need broader discussion by the community.
				\contributors\ The editors (Daniel Benarroch, Luís Brandão, Eran Tromer). 
				%%% Anyone else please contribute in reply to this post by identifying concrete suggestions of recommendations/requirements already present in the document and/or others suggested to include.
\newcol \githubissue{9}
\newcol \ccontext\ proposed in the "NIST comments on the initial ZKProof documentation" (April 06, 2019) --- item C2.
				\Chan\ ...
\newcol %\ref{...}
\rowendL
%%%%%%%%%%%%%%%%%%%%%%%%


%%%%%%%%%%%%%%%%%%%%%%%%%%%%%%%%%%%%%%%%%%%%%%%%%%%%%%%%%%%%%%%%%%%%%%%%%%%%%%%%%%%%%%%%%%%%%%%%
\newIssue{issue:explain-parameter-stat-security}{Explain the statistical security parameter} %C10
%%%%%%%%%%%%%%%%%%%%%%%%
\incItem[it:stat-sec-par:bench:security-levels]
Old sections 1.2, 1.4.3 and 2.5
\newcol \propContrib\ Discuss various examples of acceptable values of statistical security parameter, e.g., 40 bits. Explore how interactive to non-interactive transformations may affect the requirements on the statistical security parameter, e.g., making it become a computational parameter when applying Fiat-Shamir.
				\contributors\ Luís Brandão.
\newcol \githubissue{10}
\newcol \ccontext\ proposed in the "NIST comments on the initial ZKProof documentation" (April 06, 2019) --- item C19. Also discussed in the breakout session on "Interactive Zero Knowledge".
				\Chan\ ...
\newcol \ref{rev:stat-sec-par:bench:security-levels}
\rowendL
%%%%%%%%%%%%%%%%%%%%%%%%



%%%%%%%%%%%%%%%%%%%%%%%%%%%%%%%%%%%%%%%%%%%%%%%%%%%%%%%%%%%%%%%%%%%%%%%%%%%%%%%%%%%%%%%%%%%%%%%%
\newIssue{issue:clarify-scope-use-cases}{Clarify the (implicit) scope of some use-cases} %C12
%%%%%%%%%%%%%%%%%%%%%%%%
\incItem[it:clarify-scope-use-cases]
Section 3.2
\newcol \propContrib\ The last paragraph in section 2.2 says ``digital money based applications belong to the first model'' [public verifiable as a requirement]. This assertion appears implicitly scoped in a too narrow subset of conceivable applications about digital money. Conversely, one could consider a scenario where Alice wants to convince Bob, in a non-transferable way, that Alice bought something from Charlie. Consider clarifying better the scope of examples vs. the scope of areas of application.
				\contributors\ ``\href{https://github.com/DecentrilizedMan}{DecentrilizedMan}''  %%%, \needscontributors
\newcol \githubissue{12}
\newcol \ccontext\ Proposed in the "NIST comments on the initial ZKProof documentation" (April 06, 2019) --- item C15.
				\Chan\ ...
\newcol %\ref{...}
\rowendL
%%%%%%%%%%%%%%%%%%%%%%%%


%%%%%%%%%%%%%%%%%%%%%%%%%%%%%%%%%%%%%%%%%%%%%%%%%%%%%%%%%%%%%%%%%%%%%%%%%%%%%%%%%%%%%%%%%%%%%%%%
\newIssue{issue:circuits-vs-R1CS}{Compare circuits vs. R1CS} %C13
%%%%%%%%%%%%%%%%%%%%%%%%
\incItem[it:circuits-vs-R1CS]
Chapters 1 (security/theory) and 2 (implementation)
\newcol \propContrib\ The ``security/theory'' track is mentioning Boolean circuits but not R1CS. The ``implementation'' track is focused on R1CS without explaining why/when it is preferable to a circuit representation. Consider explaining better (in the ``security'' track) what is R1CS. Consider introducing and exemplifying a circuit-to-R1CS translation and/or vice-versa. Consider clarifying better in the ``implementation'' track why the focus is on R1CS, for example compared with circuits.
				\contributors\ \href{https://github.com/hasinitg}{Hasini} and \href{https://github.com/DecentrilizedMan}{DecentrilizedMan}
\newcol \githubissue{13}
\newcol \ccontext\ Proposed in the "NIST comments on the initial ZKProof documentation" (April 06, 2019) --- item C10.
				\Chan\ ...
\newcol %\ref{...}
\rowendL
%%%%%%%%%%%%%%%%%%%%%%%%





%%%%%%%%%%%%%%%%%%%%%%%%%%%%%%%%%%%%%%%%%%%%%%%%%%%%%%%%%%%%%%%%%%%%%%%%%%%%%%%%%%%%%%%%%%%%%%%%
\newIssue{issue:interactivity}{Introduction to interactive zero-knowledge proofs } %C18
%%%%%%%%%%%%%%%%%%%%%%%%
\incItem[it:interactivity:intro] 
Security section
\newcol \propContrib\ An introduction to advantages and disadvantages of interactive zero-knowledge proofs relative to non-interactive ones, and a discussion of scenarios and applications where interactive protocols may be particularly suitable or relevant.
				\contributors\ Justin Thaler and Yael Kalai (and further contributors)
\newcol \githubissue{18}, \ref{it:new-chapter-2}
\newcol \ccontext\ Discussed during the "Interactive Zero Knowledge" brekout session in the 2nd ZKProof Workshop
				\Chan\ ...
\newcol \ref{rev:interactivity-paradigm}
\rowendL
%%%%%%%%%%%%%%%%%%%%%%%%


%%%%%%%%%%%%%%%%%%%%%%%%%%%%%%%%%%%%%%%%%%%%%%%%%%%%%%%%%%%%%%%%%%%%%%%%%%%%%%%%%%%%%%%%%%%%%%%%
\newIssue{issue:distinguish-QAP-vs-PCP}{Difference between QAPs and linear PCPs} %C19
%%%%%%%%%%%%%%%%%%%%%%%%
\incItem[it:distinguish-QAP-vs-PCP] 
section 1 (security), when defining linear PCPs
\newcol \propContrib\ to outline similarities and differences between the QAP primitive and linear PCPs, and why QAPs are not just an example of a linear PCPs.
				\contributors\ Mariana Raykova
\newcol \githubissue{19}, \ref{it:new-chapter-2}
\newcol \ccontext\ Discussed at the breakout session discussing the ZKProof Community Reference documment.
				\Chan\ ...
\newcol %\ref{...}
\rowendL
%%%%%%%%%%%%%%%%%%%%%%%%


%%%%%%%%%%%%%%%%%%%%%%%%%%%%%%%%%%%%%%%%%%%%%%%%%%%%%%%%%%%%%%%%%%%%%%%%%%%%%%%%%%%%%%%%%%%%%%%%
\newIssue{issue:improve-description-apps-and-predicates}{Improve description of applications and predicates} %C20
%%%%%%%%%%%%%%%%%%%%%%%%
\incItem[it:improve-description-apps-and-predicates]
Chapter (applications)
\newcol \propContrib\ Improve the accessibility of the Applications section to meet or exceed that of Security and Implementation. This includes the following: formally expand on the existing applications for correctness and ensure that the notion of ``predicates'' is well understood.
				\contributors\ Angela Robinson and Daniel Benarroch
\newcol \githubissue{20}
\newcol \ccontext\ discussed during the breakout session about the ZKProof Community Reference document
				\Chan\ ...
\newcol %\ref{...}
\rowendL
%%%%%%%%%%%%%%%%%%%%%%%%


%%%%%%%%%%%%%%%%%%%%%%%%%%%%%%%%%%%%%%%%%%%%%%%%%%%%%%%%%%%%%%%%%%%%%%%%%%%%%%%%%%%%%%%%%%%%%%%%
\newIssue{issue:legacy-tech}{Integrate legacy technology to applications in ZK} %C21
%%%%%%%%%%%%%%%%%%%%%%%%
\incItem[it:legacy-tech]
Chapter applications
\newcol \propContrib\ Currently, there are already deployed ZKP-based solutions (e.g. anonymous credentials in Hyperledger Fabric) that might need to be bridged to others, freshly deployed, ZKP-based components (e.g. privacy-preserving payments). The goal of the bridge is to have zero or almost zero impact on the ZKP-based components already deployed. In the contribution, I want to highlight the above issues, requirements and propose how to leverage Commit-and-Prove-based techniques, like those in LegoSNARKs, to address them.
				\contributors\ Angelo de Caro
\newcol \githubissue{21}
\newcol \ccontext\ proposed during the breakout session on the ZKProof Community Reference document
				\Chan\ ...
\newcol %\ref{...}
\rowendL
%%%%%%%%%%%%%%%%%%%%%%%%


%%%%%%%%%%%%%%%%%%%%%%%%%%%%%%%%%%%%%%%%%%%%%%%%%%%%%%%%%%%%%%%%%%%%%%%%%%%%%%%%%%%%%%%%%%%%%%%%
\newIssue{issue:improve-motivation-apps}{Improve motivation in application chapter} %C22
%%%%%%%%%%%%%%%%%%%%%%%%
\incItem[it:improve-motivation-apps]
Old section 3.1
\newcol \propContrib\ Motivation for ZKPs must be improved in order to allow users to understand how ZKPs can be used to solve practical problems. In particular: Include some missing items as for example recursive composition and proof-carrying-data.
				\contributors\ Eduardo Morais
				\submit\ GitHub pull request
\newcol \githubissue{22}
\newcol \ccontext\ Breakout session: ZKProof Community Reference
				\Chan\ Included a paragraph to explain motivation for Proof Carrying Data (PCD).	
\newcol \ref{rev:app:intro:motivate}
\rowendL



%%%%%%%%%%%%%%%%%%%%%%%%%%%%%%%%%%%%%%%%%%%%%%%%%%%%%%%%%%%%%%%%%%%%%%%%%%%%%%%%%%%%%%%%%%%%%%%%
\newIssue{issue:gadgets-table}{Complete Gadgets Table} %C23
%%%%%%%%%%%%%%%%%%%%%%%%
\incItem[it:gadgets-table]
Old section 3.4
\newcol \propContrib\ Different gadgets were mentioned during the workshops. Some are already described in the document, but it is necessary to review and complete this tables.
				\contributors\ Eduardo Morais
				\submit\ GitHub pull request
\newcol \githubissue{23}
\newcol \ccontext\ Breakout session: ZKProof Community Reference
				\Chan\ Updated the \hyperref[tab:list-gadgets]{gadgets table} by filling in missing elements and making a few corrections.
				Also updated the specific tables for the following gadgets: \hyperref[tab:gadget-signature]{signature}, \hyperref[tab:gadget-encryption]{encryption}, \hyperref[tab:gadget-dist-decryption]{Distributed-decryption} and \hyperref[tab:gadget-set-membership]{set membership}				
\newcol \ref{rev:gadgets-table-fill}, \ref{rev:gadget:sig:update}, \ref{rev:gadget:enc:update}, \ref{rev:gadget:dist-enc:update}, \ref{rev:gadget:set-memb:update}
\rowendL
%%%%%%%%%%%%%%%%%%%%%%%%


%%%%%%%%%%%%%%%%%%%%%%%%%%%%%%%%%%%%%%%%%%%%%%%%%%%%%%%%%%%%%%%%%%%%%%%%%%%%%%%%%%%%%%%%%%%%%%%%
\newIssue{issue:refs-in-chapter-apps}{Include references in Application chapter} %C24
%%%%%%%%%%%%%%%%%%%%%%%%
\incItem[it:refs-in-chapter-apps]
References
\newcol \propContrib\ Some important references are missing. It is necessary to reference papers whenever relevant. See comments in version 0.1.
				\contributors\ Eduardo Morais
				\submit\ GitHub pull request
\newcol \githubissue{24}, \ref{it:legacy-tech}
\newcol \ccontext\ Breakout session: ZKProof Community Reference
				\Chan\ Added 3 references to the new paragraph (\ref{rev:app:intro:motivate}) in the introduction of the ``Applications'' chapter
\newcol \ref{rev:app:intro:new-refs}
\rowendL
%%%%%%%%%%%%%%%%%%%%%%%%








%%%%%%%%%%%%%%%%%%%%%%%%%%%%%%%%%%%%%%%%%%%%%%%%%%%%%%%%%%%%%%%%%%%%%%%%%%%%%%%%%%%%%%%%%%%%%%%%
\newIssue{issue:summaries-of-proposals}{Add description of each of the 2nd workshop proposals} %C26
%%%%%%%%%%%%%%%%%%%%%%%%
\incItem[it:summaries-of-proposals]
New appendix, where each section contains the summary of a proposal
\newcol \propContrib\ include a summary of each of the proposals and the discussions that took place, as a way to point to existing efforts of standardization
				\contributors\ Proposals' authors: Tromer, Matteo, Naure, Daniel
\newcol \githubissue{26}
\newcol \ccontext\ ...
				\Chan\ the proposals were submitted and discussed at the zkproof workshop
\newcol %\ref{...}
\rowendL
%%%%%%%%%%%%%%%%%%%%%%%%


\end{longtable} %the beginning of longtable is inserted by each new \newIssue