%%%%%%%%%%%%%%%%%%%%%%%%%%%%%%%%%%%%%%%%%%%%%%%%%%%%%%%%%%%%
%%%%%%%%%%%%%%%%%%%%%%%%%%%%%%%%%%%%%%%%%%%%%%%%%%%%%%%%%%%%
\section{Notation and Definitions}
\label{apps:notation-defs}

See \hyperref[chap:security]{Security} and \hyperref[chap:implem]{Implementation} tracks for definitions of predicate / prover / verifier / proof / proving key, etc.

When designing ZK based applications, one needs to keep in mind which of the following three models (that define the functionality of the ZKP) is needed:
\begin{enumerate}
    \item Publicly verifiable as a requirement: a scheme / use-case where the proofs are transferable, where such property is actually a requirement of the system. Only non-interactive ZK (NIZK) can actually hold this property.\luiscom{Consider revising this assertion, since if transferability is a design goal, it can still be obtained with an interactive protocol.}
    \item Designated verifier as a security feature: only the intended receiver of the proof can verify it, making the proof non-transferable.This property can apply to both interactive and non-interactive ZK.
    \item The final model is one where neither of the above is needed: a ZK where there is no need to be able to transfer but also no non-transferability requirement. Again, this model can apply both in the interactive and non-interactive model.
\end{enumerate}

For example, digital money based applications belong to the first model,\luiscom{Unclear why. Seems reasonable to devise use-cases where one wants to perform a non-transferable ZKP about something that happened with digital money.} compliance for regulation lives in the second model (albeit depending on the use-case). In general, the credential system can be in both of the last two models, given the extra constraints that would make it belong to the second model.

